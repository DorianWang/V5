%%  -*- mode: latex; mode: outline-minor; fill-column: 108 -*- 
%% Title:  lqns
%%
%% $HeadURL: http://rads-svn.sce.carleton.ca:8080/svn/lqn/trunk-V5/doc/userman/lqns.tex $
%% Original Author:     Greg Franks <greg@sce.carleton.ca>
%% Created:             12 March 2013
%%
%% ----------------------------------------------------------------------
%% $Id: lqns.tex 11342 2013-03-12 16:41:52Z greg $
%% ----------------------------------------------------------------------

\chapter{Invoking the Analytic Solver ``lqns''}
\label{sec:invoking-lqns}
The Layered Queueing Network Solver (LQNS)\index{LQNS} is used to
solved Layered Queueing Network models analytically.
\textbf{Lqns} reads its input from \texttt{filename}, specified at the
command line if present, or from the standard input otherwise.  By
default, output for an input file \texttt{filename} specified on the
command line will be placed in the file \texttt{filename.out}.  If the
\flag{p}{} switch is used, parseable output will also be written into
\texttt{filename.p}. If XML input\index{input!XML} or the \flag{x}{} switch is used, XML output\index{output!XML} will be written to 
\texttt{filename.lqxo}.  This behaviour can be changed using the
\flag{o}{}\texttt{output} switch, described below.  If several files are
named, then each is treated as a separate model and output will be
placed in separate output files.  If input is from the standard input,
output will be directed to the standard output.  The file name `\texttt{-}' is
used to specify standard input.


The \flag{o}{}\texttt{output} option can be used to direct output to the file
\texttt{output} regardless of the source of input.  Output will be XML\index{XML}\index{output!XML}
if XML input\index{XML!input} or if the \flag{x}{} switch is used, parseable output if the \flag{p}{} switch is used,
and normal output otherwise.  Multiple input files cannot be specified
when using this option.  Output can be directed to standard output by
using \flag{o}{}\texttt{-} (i.e., the output file name is `\texttt{-}'.)
\section{Command Line Options}
\label{sec:options}
\begin{description}
\item[\flag{a}{}, \longopt{ignore-advisories}]~\\
Ignore advisories.  The default is to print out all advisories.\index{advisory!ignore}
\item[\flag{b}{}, \longopt{bounds-only}]~\\
This option is used to compute the ``Type 1 throughput bounds''\index{throughput!bounds}\index{bounds!throughput} only.
These bounds are computed assuming no contention anywhere in the model
and represent the guaranteed not to exceed values.
\item[\flag{d}{}, \longopt{debug}=\emph{arg}]~\\
This option is used to enable debug output.\index{debug}
\emph{Arg} can be one of:
\begin{description}
\item[\optarg{activities}{}]
Activities -- not functional.
\item[\optarg{all}{}]
Enable all debug output.
\item[\optarg{calls}{}]
Print out the number of rendezvous between all tasks.
\item[\optarg{forks}{}]
Print out the fork-join matching process.
\item[\optarg{interlock}{}]
Print out the interlocking table and the interlocking between all tasks and processors.
\item[\optarg{joins}{}]
Joins -- not functional.
\item[\optarg{layers}{}]
Print out the contents of all of the layers found in the model.
\item[\optarg{lqx}{}]
Debug LQX parser.
\item[\optarg{overtaking}{}]
Overtaking -- not functional.
\item[\optarg{quorum}{}]
Print out results from pseudo activities used by quorum.
\item[\optarg{xml}{}]
Debug XML.
\end{description}
\item[\flag{e}{}, \longopt{error}=\emph{arg}]~\\
This option is to enable floating point exception handling.\index{floating point!exception}
\emph{Arg} must be one of the following:
\begin{enumerate}
\item \textbf{a}
Abort immediately on a floating point error (provided the floating point unit can do so).
\item \textbf{d}
Abort on floating point errors. (default)
\item \textbf{i}
Ignore floating point errors.
\item \textbf{w}
Warn on floating point errors.
\end{enumerate}
The solver checks for floating point overflow,\index{overflow} division by zero and invalid operations.
Underflow and inexact result exceptions are always ignored.


In some instances, infinities \index{infinity}\index{floating point!infinity} will be propogated within the solver.  Please refer to the
\textbf{stop-on-message-loss} pragma below.
\item[\flag{f}{}, \longopt{fast}]~\\
This option is used to set options for quick solution of a model using One-Step (Bard-Schweitzer) MVA.
It is equivalent to setting \textbf{pragma} \emph{mva}=\emph{one-step}, \emph{layering}=\emph{batched}, \emph{multiserver}=\emph{conway}
\item[\flag{H}{}, \longopt{help}=\emph{arg}]~\\
\item[\flag{I}{}, \longopt{input-format}=\emph{arg}]~\\
This option is used to force the input file format to either \emph{xml} or \emph{lqn}.
By default, if the suffix of the input filename is one of: \emph{.in}, \emph{.lqn} or \emph{.xlqn}
, then the LQN parser will be used.  Otherwise, input is assumed to be XML.
\item[\flag{n}{}, \longopt{no-execute}]~\\
Read input, but do not solve.  The input is checked for validity.  
No output is generated.
\item[\flag{o}{}, \longopt{output}=\emph{arg}]~\\
Direct analysis results to \emph{output}\index{output}.  A filename of `\texttt{-}\index{standard input}'
directs output to standard output.  If \texttt{output}is a directory, all output is saved in \texttt{output/input.out}. If the input model contains a SPEX program with loops, the SPEX output is sent to \texttt{output}; the individual model output files are found in the directory \texttt{output.d}. If \textbf{lqns} is invoked with this
option, only one input file can be specified.
\item[\flag{p}{}, \longopt{parseable}]~\\
Generate parseable output suitable as input to other programs such as
\textbf{lqn2ps(1)} and \textbf{srvndiff(1)}.  If input is from
\texttt{filename}, parseable output is directed to \texttt{filename.p}.
If standard input is used for input, then the parseable output is sent
to the standard output device.  If the \flag{o}{}\texttt{output} option is used, the
parseable output is sent to the file name \texttt{output}.
(In this case, only parseable output is emitted.)
\item[\flag{P}{}, \longopt{pragma}=\emph{arg}]~\\
Change the default solution strategy.  Refer to the PRAGMAS section\index{pragma}
below for more information.
\item[\flag{r}{}, \longopt{rtf}]~\\
Output results using Rich Text Format instead of plain text.  Processors, entries and tasks with high utilizations are coloured in red.
\item[\flag{t}{}, \longopt{trace}=\emph{arg}]~\\
This option is used to set tracing \index{tracing} options which are used to print out various
intermediate results \index{results!intermediate} while a model is being solved.
\emph{arg} can be any combination of the following:
\begin{description}
\item[\optarg{activities}{}]
Print out results of activity aggregation.
\item[\optarg{convergence}{=\emph{arg}}]
Print out convergence arg after each submodel is solved.  This option is useful for tracking the rate of convergence for a model.
The optional numeric argument supplied to this option will print out the convergence value for the specified MVA submodel, otherwise,
the convergence value for all submodels will be printed.
\item[\optarg{delta\_wait}{}]
Print out difference in entry service time after each submodel is solved.
\item[\optarg{forks}{}]
Print out overlap table for forks prior to submodel solution.
\item[\optarg{idle\_time}{}]
Print out computed idle time after each submodel is solved.
\item[\optarg{interlock}{}]
Print out interlocking adjustment before each submodel is solved.
\item[\optarg{joins}{}]
Print out computed join delay and join overlap table prior to submodel solution.
\item[\optarg{mva}{=\emph{arg}}]
Print out the MVA submodel and its solution. A numeric argument supplied to this option will print out only the specified MVA submodel, otherwise, all submodels will be printed.
\item[\optarg{overtaking}{}]
Print out overtaking calculations.
\item[\optarg{print}{}]
Print out intermediate solutions at the print interval specified in the model.  The print interval field in the input is ignored otherwise.
\item[\optarg{quorum}{}]
Print quorum traces.
\item[\optarg{throughput}{}]
Print throughput's values.
\item[\optarg{variance}{}]
Print out the variances calculated after each submodel is solved.
\item[\optarg{wait}{}]
Print waiting time for each rendezvous in the model after it has been computed.
\end{description}
\item[\flag{v}{}, \longopt{verbose}]~\\
Generate output after each iteration of the MVA solver and the convergence value at the end of each outer iteration of the solver.
\item[\flag{V}{}, \longopt{version}]~\\
Print out version and copyright information.\index{version}\index{copyright}
\item[\flag{w}{}, \longopt{no-warnings}]~\\
Ignore warnings.  The default is to print out all warnings.\index{warning!ignore}
\item[\flag{x}{}, \longopt{xml}]~\\
Generate XML output regardless of input format.
\item[\flag{z}{}, \longopt{special}=\emph{arg}]~\\
This option is used to select special options.  Arguments of the form
\emph{nn} are integers while arguments of the form \emph{nn.n} are real
numbers.  \emph{Arg} can be any of the following:
\begin{description}
\item[\optarg{convergence-value}{=\emph{arg}}]
Set the convergence value to \emph{arg}.  
\emph{Arg} must be a number between 0.0 and 1.0.
\item[\optarg{full-reinitialize}{}]
For multiple runs, reinitialize all processors.
\item[\optarg{generate}{=\emph{arg}}]
This option is used for debugging the solver.  A directory named \emph{arg} will be created containing source code for invoking the MVA solver directly.
\item[\optarg{ignore-overhanging-threads}{}]
Ignore the effect of the overhanging threads.
\item[\optarg{iteration-limit}{=\emph{arg}}]
Set the maximum number of iterations to \emph{arg}.
\emph{Arg} must be an integer greater than 0.  The default value is 50.
\item[\optarg{man}{=\emph{arg}}]
Output this manual page.  If an optional \emph{arg}
is supplied, output will be written to the file named \emph{arg}.
Otherwise, output is sent to stdout.
\item[\optarg{min-steps}{=\emph{arg}}]
Force the solver to iterate min-steps times.
\item[\optarg{mol-ms-underrelaxation}{=\emph{arg}}]
Set the under-relaxation factor to \emph{arg} for the MOL multiserver approximation.
\emph{Arg} must be a number between 0.0 and 1.0.
The default value is 0.5.\item[\optarg{overtaking}{}]
Print out overtaking probabilities.\item[\optarg{print-interval}{=\emph{arg}}]
Set the printing interval to \emph{arg}.  The
\flag{d}{} or \flag{v}{} options must also be selected to display intermediate results.
The default value is 10.
\item[\optarg{single-step}{}]
Stop after each MVA submodel is solved.  Any character typed at the terminal except end-of-file will resume the calculation.  End-of-file will cancel single-stepping altogether.
\item[\optarg{skip-layer}{=\emph{arg}}]
Ignore submodel \emph{arg} during solution.
\item[\optarg{tex}{=\emph{arg}}]
Output this manual page in LaTeX format.   If an optional \emph{arg}
is supplied, output will be written to the file named \emph{arg}.
Otherwise, output is sent to stdout.
\item[\optarg{underrelaxation}{=\emph{arg}}]
Set the underrelaxation to \emph{arg}.
\emph{Arg} must be a number between 0.0 and 1.0.
The default value is 0.9.
\end{description}
If any one of \emph{convergence}, \emph{iteration-limit}, or\emph{print-interval} are used as arguments, the corresponding 
value specified in the input file for general information, `G', is
ignored.  
\item[\flag{c}{}, \longopt{convergence}=\emph{arg}]~\\
Set the convergence value to \emph{arg}.  
\emph{Arg} must be a number between 0.0 and 1.0.
\index{convergence!value}\item[\flag{e}{}, \longopt{exact-mva}]~\\
Use Exact MVA to solve all submodels.\index{MVA!exact}
\item[\longopt{hwsw-layering}]~\\
\item[\flag{i}{}, \longopt{iteration-limit}=\emph{arg}]~\\
Set the maximum number of iterations to \emph{arg}.
\emph{Arg} must be an integer greater than 0.  The default value is 50.
\item[\longopt{srvn-layering}]~\\
Solve the model using submodels containing exactly one server.
\item[\longopt{squashed-layering}]~\\
Use only one submodel to solve the model.
\item[\flag{m}{}, \longopt{method-of-layers}]~\\
This option is to use the Method Of Layers solution approach to solving the layer submodels.
\item[\longopt{processor-sharing}]~\\
Use Processor Sharing scheduling at all fixed-rate processors.
\item[\flag{s}{}, \longopt{schweitzer-amva}]~\\
Use Bard-Schweitzer approximate MVA to solve all submodels.\index{MVA!Bard-Schweitzer}
\item[\flag{o}{}, \longopt{no-stop-on-message-loss}]~\\
Do not stop the solver on overflow (infinities) for open arrivals or send-no-reply messages to entries.  The default is to stop with an
error message indicating that the arrival rate is too high for the service time of the entry
\item[\longopt{trace-mva}]~\\
Output the inputs and results of each MVA submodel for every iteration of the solver.
\item[\flag{u}{}, \longopt{underrelaxation}=\emph{arg}]~\\
Set the underrelaxation to \emph{arg}.
\emph{Arg} must be a number between 0.0 and 1.0.
The default value is 0.9.
\item[\longopt{no-variance}]~\\
Do not use variances in the waiting time calculations.
\item[\longopt{reload-lqx}]~\\
Re-run the LQX\index{LQX} program without re-solving the models.  Results must exist from a previous solution run.
This option is useful if LQX print statements are changed.
\item[\longopt{no-header}]~\\
\item[\longopt{debug-lqx}]~\\
Output debugging information as an LQX\index{LQX!debug} program is being parsed.
\item[\longopt{debug-xml}]~\\
Output XML\index{XML!debug} elements and attributes as they are being parsed.   Since the XML parser usually stops when it encounters an error,
this option can be used to localize the error.
\item[\longopt{debug-srvn}]~\\
\end{description}


\textbf{Lqns} exits\index{exit!success} with 0 on success, 1 if the model failed to converge,\index{convergence!failure}
2 if the input was invalid\index{input!invalid}, 4 if a command line argument was\index{command line!incorrect}
incorrect, 8 for file read/write problems and -1 for fatal errors\index{error!fatal}.  If
multiple input files are being processed, the exit code is the
bit-wise OR of the above conditions.
\section{Pragmas}
\label{sec:lqns-pragmas}
\emph{Pragmas}\index{pragma} are used to alter the behaviour of the solver in a
variety of ways.  They can be specified in the input file with
``\#pragma'', on the command line with the \flag{P}{} option, or through
the environment variable \index{environment variable}\emph{LQNS\_PRAGMAS}\index{LQNS\_PRAGMAS@\texttt{LQNS\_PRAGMAS}}.  Command line
specification of pragmas overrides those defined in the environment
variable which in turn override those defined in the input file.  The
following pragmas are supported.  Invalid pragma\index{pragma!invalid} specification at the
command line will stop the solver.  Invalid pragmas defined in the
environment variable or in the input file are ignored as they might be
used by other solvers.
\begin{description}
\item[\optarg{cycles}{=\emph{arg}}]~\\
This pragma is used to enable or disable cycle detection\index{cycle!detection} in the call
graph.\index{call graph}  Cycles may indicate the presence of deadlocks.\index{deadlock}
\emph{Arg} must be one of: 
\begin{description}
\item[\optarg{allow}{}]
Allow cycles in the call graph.  The interlock\index{interlock} adjustment is disabled.
\item[\optarg{disallow}{}]
Disallow cycles in the call graph.
\end{description}
The default is disallow.
\item[\optarg{interlocking}{=\emph{arg}}]~\\
The interlocking\index{interlock} is used to correct the throughputs\index{throughput!interlock} at stations as a
result of solving the model using layers~\cite{perf:franks-95-ipds-interlock}.  This pragma is used to
choose the algorithm used.
\emph{Arg} must be one of: 
\begin{description}
\item[\optarg{none}{}]
Do not perform interlock adjustment.
\item[\optarg{throughput}{}]
Perform interlocking by adjusting throughputs.
\end{description}
The default is throughput.
\item[\optarg{layering}{=\emph{arg}}]~\\
This pragma is used to select the layering strategy\index{layering!strategy} used by the solver.
\emph{Arg} must be one of: 
\begin{description}
\item[\optarg{batched}{}]
Batched layering\index{batched layers}\index{layering!batched} -- solve layers composed of as many servers as possible from top to bottom.
\item[\optarg{batched-back}{}]
Batched layering with back propagation -- solve layers composed of as many servers as possible from top to bottom, then from bottom to top to improve solution speed.
\item[\optarg{hwsw}{}]
Hardware/software layers\index{hardware-software layers}\index{layers!hardware-software} -- The model is solved using two submodels:
One consisting solely of the tasks in the model, and the other with the tasks calling the processors.
\item[\optarg{mol}{}]
Method Of layers\index{method of layers}\index{layering!method of} -- solve layers using the Method of Layers~\cite{perf:rolia-95-ieeese-mol}\index{Method of Layers}\index{layering!Method of Layers}. Layer spanning is performed by allowing clients to appear in more than one layer.
\item[\optarg{mol-back}{}]
Method Of layers -- solve layers using the Method of Layers.  Software submodels are solved top-down then bottom up to improve solution speed.
\item[\optarg{squashed}{}]
Squashed layers\index{squashed layers}\index{layering!squashed} -- All the tasks and processors are placed into one submodel.
Solution speed may suffer because this method generates the most number of chains in the MVA solution.  See also \flag{P}{}\emph{mva}.
\item[\optarg{srvn}{}]
SRVN layers\index{srvn layers}\index{layering!srvn} -- solve layers composed of only one server.
This method of solution is comparable to the technique used by the \textbf{srvn} solver.  See also \flag{P}{}\emph{mva}.
\end{description}
The default is batched-back.
\item[\optarg{multiserver}{=\emph{arg}}]~\\
This pragma is used to choose the algorithm for solving multiservers\index{multiserver!algorithm}.
\emph{Arg} must be one of: 
\begin{description}
\item[\optarg{bruell}{}]
Use the Bruell multiserver\index{multiserver!Bruell}~\cite{queue:bruell-84-peva-load-dependent} calculation for all multiservers.
\item[\optarg{conway}{}]
Use the Conway multiserver\index{multiserver!Conway}~\cite{queue:deSouzaeSilva-87,queue:conway-88} calculation for all multiservers.
\item[\optarg{reiser}{}]
Use the Reiser multiserver\index{multiserver!Reiser}~\cite{queue:reiser-79} calculation for all multiservers.
\item[\optarg{reiser-ps}{}]
Use the Reiser multiserver calculation for all multiservers. For multiservers with multiple entries, scheduling is processor sharing\index{processor!sharing}, not FIFO. 
\item[\optarg{rolia}{}]
Use the Rolia\index{multiserver!Rolia}~\cite{perf:rolia-92,perf:rolia-95-ieeese-mol} multiserver calculation for all multiservers.
\item[\optarg{rolia-ps}{}]
Use the Rolia multiserver calculation for all multiservers. For multiservers with multiple entries, scheduling is processor sharing\index{processor!sharing}, not FIFO. 
\item[\optarg{schmidt}{}]
Use the Schmidt multiserver\index{multiserver!Schmidt}~\cite{queue:schmidt-97} calculation for all multiservers.
\item[\optarg{suri}{}]
experimental.
\end{description}
The default multiserver\index{multiserver!default} calculation uses the the Conway multiserver for multiservers with less than five servers, and the Rolia multiserver otherwise.

\item[\optarg{mva}{=\emph{arg}}]~\\
This pragma is used to choose the MVA\index{MVA} algorithm used to solve the submodels.
\emph{Arg} must be one of: 
\begin{description}
\item[\optarg{exact}{}]
Exact MVA\index{MVA!exact}.  Not suitable for large systems.
\item[\optarg{fast}{}]
Fast Linearizer
\item[\optarg{linearizer}{}]
Linearizer.\index{MVA!Linearizer}\index{Linearizer}
\item[\optarg{one-step}{}]
Perform one step of Bard Schweitzer approximate MVA for each iteration of a submodel.  The default is to perform Bard Schweitzer approximate MVA until convergence for each submodel.  This option, combined with \flag{P}{}\emph{layering=srvn} most closely approximates the solution technique used by the \textbf{srvn} solver.
\item[\optarg{one-step-linearizer}{}]
Perform one step of Linearizer approximate MVA for each iteration of a submodel.  The default is to perform Linearizer approximate MVA until convergence for each submodel.
\item[\optarg{schweitzer}{}]
Bard-Schweitzer approximate MVA.\index{MVA!Bard-Schweitzer}\index{Bard-Schweitzer}
\end{description}
The default is linearizer.
\item[\optarg{overtaking}{=\emph{arg}}]~\\
This pragma is usesd to choose the overtaking\index{overtaking} approximation.
\emph{Arg} must be one of: 
\begin{description}
\item[\optarg{markov}{}]
Markov phase 2 calculation.\index{overtaking!Markov}
\item[\optarg{none}{}]
Disable all second phase servers.  All stations are modeled as having a single phase by summing the phase information.
\item[\optarg{rolia}{}]
Use the method from the Method of Layers.\index{overtaking!Method of Layers}
\item[\optarg{simple}{}]
Simpler, but faster approximation.
\item[\optarg{special}{}]
?
\end{description}
The default is rolia.
\item[\optarg{processor}{=\emph{arg}}]~\\
Force the scheduling type\index{scheduling!processor}\index{processor!scheduling} of all uni-processors to the type specfied.
\begin{description}
\item[\optarg{fcfs}{}]
All uni-processors are scheduled first-come, first-served.
\item[\optarg{hol}{}]
All uni-processors are scheduled using head-of-line priority.\index{priority!head of line}
\item[\optarg{ppr}{}]
All uni-processors are scheduled using priority, pre-emptive resume.\index{priority!preemptive-resume}
\item[\optarg{ps}{}]
All uni-processors are scheduled using processor sharing.\index{processor sharing}
\end{description}
The default is to use the processor scheduling specified in the model.

\item[\optarg{severity-level}{=\emph{arg}}]~\\
This pragma is used to enable or disable warning messages.
\begin{description}
\item[\optarg{advisory}{}]
\item[\optarg{all}{}]
\item[\optarg{run-time}{}]
\item[\optarg{warning}{}]
\end{description}
The default is all.
\item[\optarg{stop-on-message-loss}{=\emph{arg}}]~\\
This pragma is used to control the operation of the solver when the
arrival rate\index{arrival rate} exceeds the service rate of a server.
\emph{Arg} must be one of: 
\begin{description}
\item[\optarg{false}{}]
Ignore queue overflows\index{overflow} for open arrivals\index{open arrival!overflow} and send-no-reply\index{send-no-reply!overflow} requests.  If a queue overflows, its waiting times is reported as infinite.\index{infinity}\item[\optarg{true}{}]
Stop if messages are lost.
\end{description}
The default is false.
\item[\optarg{tau}{=\emph{arg}}]~\\
Set the tau adjustment factor to \emph{arg}.
\emph{Arg} must be an integer between 0 and 25.
A value of \emph{zero} disables the adjustment.
\item[\optarg{threads}{=\emph{arg}}]~\\
This pragma is used to change the behaviour of the solver when solving
models with fork-join\index{fork}\index{join} interactions.
\begin{description}
\item[\optarg{exponential}{}]
Use exponential values instead of three-point approximations in all approximations\index{three-point approximation}~\cite{perf:jiang-96}.
\item[\optarg{hyper}{}]
Inflate overlap probabilities based on arrival instant estimates.
\item[\optarg{mak}{}]
Use Mak-Lundstrom\index{Mak-Lundstrom}~\cite{perf:mak-90} approximations for join delays.\index{join!delay}
\item[\optarg{none}{}]
Do not perform overlap calculation for forks.\index{overlap calculation}
\end{description}
The default is hyper.
\item[\optarg{variance}{=\emph{arg}}]~\\
This pragma is used to choose the variance\index{variance} calculation used by the solver.
\begin{description}
\item[\optarg{init-only}{}]
Initialize the variances, but don't recompute as the model is solved.\index{variance!initialize only}
\item[\optarg{mol}{}]
Use the MOL variance calculation.\index{variance!Method of Layers}\index{Method of Layers!variance}
\item[\optarg{no-entry}{}]
By default, any task with more than one entry will use the variance calculation.  This pragma will switch off the variance calculation for tasks with only one entry.
\item[\optarg{none}{}]
Disable variance adjustment.  All stations in the MVA submodels are either delay- or FIFO-servers.
\item[\optarg{stochastic}{}]
?
\end{description}

\end{description}
\section{Stopping Criteria}
\label{sec:lqns-stopping-criteria}
\textbf{Lqns} computes the model results by iterating through a set of
submodels until either convergence\index{convergence} is achieved, or the iteration limit\index{iteration limit|textbf}
is hit. Convergence is determined by taking the root of the mean of
the squares of the difference in the utilization of all of the servers
from the last two iterations of the MVA solver over the all of the
submodels then comparing the result to the convergence value specified
in the input file. If the RMS change in utilization is less than
convergence value\index{convergence!value|textbf}, then the results are considered valid.


If the model fails to converge,\index{convergence!failure} three options are available:
\begin{enumerate}
\item reduce the under-relaxation coefficient. Waiting and idle times are
propogated between submodels during each iteration. The
under-relaxation coefficient determines the amount a service time is
changed between each iteration. A typical value is 0.7 - 0.9; reducing
it to 0.1 may help.
\item increase the iteration limit.\index{iteration limit} The iteration limit sets the upper bound
on the number of times all of the submodels are solved. This value may
have to be increased, especially if the under-relaxation coefficient
is small, or if the model is deeply nested. The default value is 50
iterations.
\item increase the convergence test value\index{convergence!value}. Note that the convergence value
is the standard deviation in the change in the utilization of the
servers, so a value greater than 1.0 makes no sense.
\end{enumerate}


The convergence value can be observed using \flag{t}{}\emph{convergence} flag.
\section{Model Limits}
\label{sec:model-limits}
The following table lists the acceptable parameter types for
\textbf{lqns}.  An error will
be reported if an unsupported parameter is supplied except when the
value supplied is the same as the default.


%%--------------------------------------------------------------------
%% Table Begin
%%--------------------------------------------------------------------
\begin{table}[htbp]
  \centering
  \begin{tabular}[c]{ll}
    Parameter&lqns \\
    \hline
    Phases\index{ phase!maximum} & 3\\
    Scheduling\index{ scheduling} & FIFO, HOL, PPR\\
    Open arrivals\index{ open arrival} & yes\\
    Phase type\index{ phase!type} & stochastic, deterministic\\
    Think Time\index{ think time} & yes\\
    Coefficient of variation\index{ coefficient of variation} & yes\\
    Interprocessor-delay\index{ interprocessor delay} & yes\\
    Asynchronous connections\index{ asynchronous connections} & yes\\
    Forwarding\index{ forwarding} & yes\\
    Multi-servers\index{ multi-server} & yes\\
    Infinite-servers\index{ infinite server} & yes\\
    Max Entries\index{ entry!maximum} & 1000\\
    Max Tasks\index{ task!maximum} & 1000\\
    Max Processors\index{ processor!maximum} & 1000\\
    Max Entries per Task & 1000\\
    \hline
  \end{tabular}
  \caption{\label{tab:lqns-model-limits}LQNS Model Limits\index{limits!lqns}.}
\end{table}
\section{Diagnostics}
\label{sec:lqns-diagnostics}
Most diagnostic messages result from errors in the input file.
If the solver reports errors, then no solution will be generated for
the model being solved.  Models which generate warnings may not be
correct.  However, the solver will generate output.


Sometimes the model fails to converge\index{convergence!failure}, particularly if there are several
heavily utilized servers in a submodel.  Sometimes, this problem can
be solved by reducing the value of the under-relaxation coefficient.  It
may also be necessary to increase the iteration-limit\index{iteration limit}, particularly if
there are many submodels.  With replicated models, it may be necessary
to use `srvn' layering to get the model to converge.  Convergence can be tracked
using the \flag{t}{}\emph{convergence} option.


The solver will sometimes report some servers with `high' utilization.
This problem is the result of some of the approximations used, in particular, two-phase servers.
Utilizations in excess of 10\% are likely the result of failures in the solver.
Please send us the model file so that we can improve the algorithms.
%%% Local Variables: 
%%% mode: latex
%%% mode: outline-minor 
%%% fill-column: 108
%%% TeX-master: "userman"
%%% End: 
