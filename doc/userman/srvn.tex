%% -*- mode: latex; mode: outline-minor; fill-column: 108 -*-
%%
%% Title:  spex
%%
%% $HeadURL: http://franks.dnsalias.com/svn/lqn/trunk/doc/userman/lqns.tex $
%% Original Author:     Alex Hubbard
%% Created:             27 April 2011
%%
%% ------------------------------------------------------------------------
%%  $Id: srvn.tex 11987 2014-04-16 20:57:40Z greg $
%% ------------------------------------------------------------------------
%%
\lstdefinelanguage{LQN}{
  basicstyle=\ttfamily,
  keywords={G,P,U,T,E,A,R,C,p,g,t,s,c,f,y,z,u,F,-1,->,\#pragma},
  classoffset=1,
  morekeywords={<param>,<proc-id>,<group-id>,<task-id>,<entry-id>,<activity-id>,<sched>,<opt-mult>,<opt-cap>,<opt-pri>,<opt-grp>,<opt-think-time>,<opt-obs>,<opt-repl>,<value>,<expression>,<entry-list>,<activity-list>,<expression-list>,<real>,<int>,<string>},
  keywordstyle=\itshape,
  classoffset=0,
  alsoletter={-1<>},
  sensitive=true,
  morecomment=[l]{\# },
  morestring=[b]'',
  index={pragma}
}
\newcommand{\nonterminal}[1]{$<$\emph{#1}$>$\mbox{}}
%%
%%
%%
\chapter{LQN Input File Format}
\label{ch:srvn}

This Chapter describes the original `SRVN' input file format, augmented with the Software Performance
EXperiment driver (SPEX)\index{SPEX} grammar.  In this model format models are specified breadth-first, in
contrast to the XML format described in \S\ref{sec:xml-grammar} where models are specified depth-first.
This specification means that all resources such as processors, tasks and entries, are defined before they
are referenced. Furthermore, each resource is grouped into its own section in the input file.
Figure~\ref{fig:srvn-schema} shows the basic schema and Listing~\ref{lst:lqn-file-layout} shows the basic
layout of the model file.

\begin{figure}[ht]
  \centering
  \epsffile{srvn-schema/srvn-schema.eps}
  \caption{SRVN input schema}
  \label{fig:srvn-schema}
\end{figure}

\lstset{language=LQN,numbersep=10pt,firstnumber=1,texcl}
\begin{lstlisting}[float,caption={LQN file layout},label=lst:lqn-file-layout]
  # \color{blue}Pragmas
  #pragma <param>=<value>

  # \color{red}Parameters (SPEX)
  $var = <expression>
  $var = [ <expression-list> ]

  # \color{blue}General Information
  G "<string>" <real> <int> <int> <real> -1

  # \color{black}Processor definitions
  P 0
    p <proc-id> <sched> <opt-mult> <opt-repl> <opt-obs>
  -1

  # \color{blue}Group definitions
  U 0
    g <group-id> <real> <opt-cap> <proc-id>
  -1

  # \color{black}Task definitions
  T 0
    t <task-id> <sched> <entry-list> -1 <proc-id> <opt-pri> <opt-think-time>
                <opt-mult> <opt-repl> <opt-grp> <opt-obs>
  -1

  # Entry definitions
  E 0
    A <activity-id>
    s <entry-id> <real> ... -1 <opt-obs>
    y <entry-id> <entry-id> <real> ... -1 <opt-obs>
  -1

  # \color{blue}Activity definitions
  A <task-id>
    s <activity-id> <real> <opt-obs>
    y <activity-id> <entry-id> <real> <opt-obs>
  :
    <activity-list> -> <activity-list>
  -1

  # \color{red}Result defintions (SPEX)
  R 0
    $var = <expression>
  -1

  # \color{red}Convergence defintions (SPEX)
  C 0
    $var = <expression>
  -1
\end{lstlisting}

Each of the sections within the input file begins with a key-letter, as follows:
\begin{description}
\item[\$] SPEX parameters\index{SPEX!parameters} (optional).
\item[G] General solver parameters (optional).
\item[P] Processor definitions.
\item[U] Processor group definitions (optional).
\item[T] Task definitions.
\item[E] Entry definitions.
\item[A] Task activity definitions  (optional).
\item[R] SPEX result definitions\index{SPEX!results}  (optional).
\item[C] SPEX convergence\index{SPEX!convergence}  (optional).
\end{description}
Section~\ref{sec:lqn-file-format} describes the input sections necessary to solve a model, i.e.\ \texttt{P},
\texttt{U} \texttt{T}, \texttt{E}, and \texttt{A}.  Section~\ref{sec:spex} describes the additional input
sections for solving multiple models using SPEX, i.e.\ \texttt{\$}, \texttt{R}, and \texttt{C}.  The
complete input grammar is listed in Appendix~\ref{sec:old-grammar}.

\section{Lexical Conventions}

The section describes the lexical conventions of the SRVN input file format.

\subsection{White Space}

White space\dindex{LQN}{white space}, such as spaces, tabs and new-lines, is ignored except within strings.
Object definitions can span multiple lines.

\subsection{Comments}

Any characters following a hash mark (\#)\index{\#} through to the end of the line are considered to be a
comment\dindex{LQN}{comment} and are generally ignored.  However, should a line begin with optional whitespace
followed by `\texttt{\#pragma}'\index{\#pragma}\index{pragma}, then the remainder of the line will be
treated by the solver as a pragma (more on pragmas below).

\subsection{Identifiers}
\label{sec:lqn-identifiers}\label{LQN!identifiers|textbf}

Identifiers\dindex{LQN}{identifier} are used to name the objects in the model.  They consist of zero or more
leading underscores (`\_'), followed by a character, followed by any number of characters, numbers or
underscores. Punctuation characters and other special characters such as the dollar-sign (`\texttt{\$}') are
not permitted.  Non-numeric identifiers must be a minimum of two characters in length\footnote{Single
  characters are used for section and record keys.}  The following, {\tt 1}, {\tt p1}, {\tt p\_1}, and {\tt
  \_\_P\_21\_proc} are valid identifiers, while {\tt \$proc} and {\tt \$1} are not.

\subsection{Variables}
\label{sec:lqn-variables}\label{LQN!variables|textbf}

Variables \dindex{LQN}{parameter} are used to set values of various objects such as the multiplicity of
tasks and the service times of the phases of entries.  Variables are modifed by
SPEX\dindex{SPEX}{Variables} (see~\S\ref{sec:spex}) to run multiple experiments.  Variables start with a
dollar-sign (`\texttt{\$}') followed by any number of characters, numbers or underscores.  {\tt \$var} and
{\tt \$1} are valid variables while {\tt \$\$} is not.

\section{LQN Model Specification}
\label{sec:lqn-file-format}
\index{LQN|(textbf}

This section describes the mandatory and option input for a basic LQN model file.  SPEX information, namely
\emph{Variables}, (\S\ref{sec:lqn-activity-information}), \emph{Report Information}
(\S\ref{sec:lqn-report-information}) and \emph{Convergence Information}
(\S\ref{sec:lqn-convergence-information}) are described in the section that follows.  All input files are
composed of three mandatory sections: \emph{Processor Information} (\S\ref{sec:lqn-processor-information}),
\emph{Task Information} (\S\ref{sec:lqn-task-information}) and \emph{Entry Information}
(\S\ref{sec:lqn-entry-information}), which define the processors, tasks and entries respectively in the
model.  All of the other sections for a basic model file are optional.  They are: \emph{Pragmas},
\emph{General Information} (\S\ref{sec:lqn-general-information}), \emph{Group Information}
(\S\ref{sec:lqn-group-information}), and \emph{Activity Information}.  The syntax of these specifications
are described next in the order in which they appear in the input model.

\subsection{Pragmas}

Any line beginning with optional whitespace followed by the word `\texttt{\#pragma}'\dindex{LQN!pragma}
defines a pragma which is used by either the analytic solver or the simulator to change its behaviour.  The
syntax for a pramga directive is shown in line~2 in Listing~\ref{lst:lqn-file-layout}.  Pragma's which are
not handled by either the simulator or the analytic solver are ignored.  Pragma's can appear anywhere in the
input file\footnote{Pragma's are processed during lexical analysis.}  though they typically appear first.

\subsection{General Information}
\label{sec:lqn-general-information}

The optional general information section is used to set various control parameters for the analytic solver
LQNS\index{LQNS!parameters}.  These parameters, with the exception of the model comment, are ignored by the
simulator, lqsim.  Listing~\ref{lst:lqn-general-info} shows the format of this section.  Note that these
parameters can also be set using SPEX variables, described below
in~\S\ref{sec:spex-control}\index{SPEX!parameter!control}.

\lstset{language=LQN,basicstyle=\ttfamily,numbersep=10pt,firstnumber=1}
\begin{lstlisting}[caption={General Information},label=lst:lqn-general-info,frame=single]
G "<string>"	# Model title.
  <real> 	# convergence value
  <int>         # iteration limit
  <int>         # Optional print interval.
  <real>        # Optional under-relaxation.
-1
\end{lstlisting}

\subsection{Processor Information}
\label{sec:lqn-processor-information}

Processors\dindex{LQN}{processor} are specified in the processor information section of the input file using
the syntax shown in Listing~\ref{lst:lqn-proc-info}.  The start of the section is identified using
``\texttt{P} \nonterminal{int}'' and ends with ``\texttt{-1}''.  The \nonterminal{int} parameter is either
the number of processor definitions in this section, or zero\footnote{The number of processors,
  \nonterminal{int}, is ignored with all current solvers.}.

\lstset{language=LQN,basicstyle=\ttfamily,numbersep=10pt,firstnumber=1}
\begin{lstlisting}[caption={Processor Information},label=lst:lqn-proc-info,frame=single]
P <int>
  p <proc-id> <sched> <opt-mult>
-1
\end{lstlisting}

Each processor in the model is defined using the syntax shown in line~2 in Listing~\ref{lst:lqn-proc-info}.
Each record in this section beginning with a `\texttt{p}' defines a processor. \nonterminal{proc-id} is
either an integer or an identifier (defined earlier in \S\ref{sec:lqn-identifiers}).  \nonterminal{sched} is
used to define the scheduling discipline for the processor and is one of the code letters listed in
Table~\ref{tab:lqn-proc-sched}.  The scheduling disciplines supported by the model are described in
Section~\ref{sec:processors}.  Finally, the optional \nonterminal{opt-mult} specifies the number of copies
of this processor serving a common queue.  Multiplicity is specified using the syntax shown in
Table~\ref{tab:lqn-multiplicity}.  By default, a single copy of a processor is used for the model.

\begin{table}[h]
  \centering
  \begin{tabular}{|c|l|}
    \hline
    \textbf{\nonterminal{sched}} & \multicolumn{1}{c|}{\textbf{Scheduling Discipline}} \\
    \hline
    \texttt{f} & First-come, first served\index{scheduling!FCFS}.\\
    \texttt{p} & Priority-preemptive resume.\\
    \texttt{r} & Random.\\
    \texttt{i} & Delay (infinite server).\\
    \texttt{h} & Head-of-Line\index{scheduling!head-of-line}.\\
    \texttt{c} \nonterminal{real} & Completely fair share with time quantum  \nonterminal{real}.\index{scheduling!completely fair}\\
    \texttt{s} \nonterminal{real} & Round Robin with time quantum \nonterminal{real}.\\
    \hline
  \end{tabular}
  \caption{Processor Scheduling Disciplines (see~\S\protect\ref{sec:processors}).\dindex{processor}{scheduling}}
  \label{tab:lqn-proc-sched}
\end{table}

\begin{table}[h]
  \centering
  \begin{tabular}{|c|l|}
    \hline
    \textbf{\nonterminal{opt-mult}} & \multicolumn{1}{c|}{\textbf{Multiplicity}} \\
    \hline
    \texttt{m} \nonterminal{int} & \nonterminal{int} identical copies with a common queue. \\
    \texttt{i} & Infinite (or delay).\\
    \hline
    \hline
    \textbf{\nonterminal{opt-repl}} & \multicolumn{1}{c|}{\textbf{Repliplication}} \\
    \hline
    \texttt{r} \nonterminal{int} & \nonterminal{int} replicated copies with separate queues. \\
    \hline
  \end{tabular}
  \caption{Multiplicity and Replication (see~\S\protect\ref{sec:replication}).\dindex{LQN}{multiplicity}}
  \label{tab:lqn-multiplicity}
\end{table}

\subsection{Group Information}
\label{sec:lqn-group-information}

Groups\dindex{LQN}{group} are specified in the group information section of

\begin{lstlisting}[caption={Group Information},label=lst:lqn-group-info,frame=single,firstnumber=1]
U <int>
  g <group-id> <real> <opt-cap> <proc-id>
-1
\end{lstlisting}

\subsection{Task Information}
\label{sec:lqn-task-information}

Tasks\dindex{LQN}{task} are specified in the task information section of the input file using the syntax
shown in Listing~\ref{lst:lqn-task-info}.  The start of the task section is identified using
``\texttt{T}~\nonterminal{int}'' and ends with ``\texttt{-1}''.  The \nonterminal{int} parameter is either
the number of task definitions in this section, or zero.

\begin{lstlisting}[caption={Task Information},label=lst:lqn-task-info,frame=single,firstnumber=1]
T <int>
  t <task-id> <sched> <entry-list> -1 <opt-queue-length> <opt-tokens>  
              <proc-id> <opt-pri> <opt-think-time>
              <opt-mult> <opt-repl> <opt-grp>
  I <task-id> <task-id> <int>    # fan-in for replication
  O <task-id> <task-id> <int>    # fan-out for replication
-1
\end{lstlisting}

Each task definition within this section starts with a `\texttt{t}' and is is defined using the syntax shown
in lines~2 and 3 of Listing~\ref{lst:lqn-task-info}\footnote{Line 3 is a continuation of line 2.}.
\nonterminal{task-id} is an identifier which names the task.  \nonterminal{sched} is used to define the
request distribution for reference tasks\dindex{reference}{task}, or the scheduling discipline for
non-reference tasks.  The scheduling and distribution code letters are shown in
Table~\ref{tab:lqn-task-sched}.  Some disciplines are only supported by the simulator; these are identified
using ``\dag''.  \nonterminal{entry-list} is a list of idententifiers naming the entries of the task.  The
optional \nonterminal{opt-pri} is used to set the priority for the task provided that the processor running
the task is scheduled using a priority discipline.  The optional \nonterminal{opt-think-time} specifies a
think time\index{think time} for a reference task.  The optional \nonterminal{opt-mult} specifies the number
of copies of this task serving a common queue.  Multiplicity is specified using the syntax shown in
Table~\ref{tab:lqn-multiplicity}.  By default, a single copy of a task is used for the model. Finally, the
optional \nonterminal{opt-grp} is used to identify the group that this task belongs to provided that the
task's processor is using fair-share scheduling\index{scheduling!fair share}


\begin{table}[h]
  \centering
  \begin{tabular}{|c|l|}
    \hline
    \multicolumn{2}{|c|}{Reference tasks (customers)\index{task!reference}\index{reference task}.}\\
    \hline
    \textbf{\nonterminal{sched}} & \multicolumn{1}{c|}{\textbf{Request Distribution}} \\
    \hline
    r & Poisson.\\
    b & Bursty\dag.\\
    u & Uniform\dag.\\
    \hline
    \hline
    \multicolumn{2}{|c|}{Non-Reference tasks (servers)\index{task!reference}\index{reference task}.}\\
    \hline
    \textbf{\nonterminal{sched}} & \multicolumn{1}{c|}{\textbf{Queueing Discipline}} \\
    \hline
    n & First come, first served\index{scheduling!FCFS}.\\
    P & Polled service at entries\dag.\\
    h & Head-of-line priority\index{scheduling!head-of-line}.\\
    f & First come, first served\index{scheduling!FCFS}.\\
    i & Infinite (delay) server.\\
    w & Read-Write lock task\dag.\\
    S & Semaphore task\dag.\\
    \hline
  \end{tabular}
  \caption{Task Scheduling Disciplines (see~\S\protect\ref{sec:tasks}).}
  \label{tab:lqn-task-sched}
\end{table}

\begin{table}
  \centering
  \begin{tabular}{|c|l|}
    \hline
    \textbf{Option} & \multicolumn{1}{c|}{\textbf{Parameter}} \\
    \hline
    \nonterminal{integer} & Task priority for tasks running on processors supporting priorities.\\
    \texttt{z} \nonterminal{real} & Think Time for reference tasks.\\
    \texttt{q} \nonterminal{real} & Maximum queue length for asynchronous requests. \\
    \texttt{T} \nonterminal{integer} & Initial tokens at semaphore task\dag.\\
    \texttt{m} \nonterminal{integer} & Task multiplicity.\\
    \texttt{r} \nonterminal{integer} & Task replication.\\
    \texttt{g} \nonterminal{identifier} & Group identifier for tasks running of processors with fair share scheduling.\\
    \hline
  \end{tabular}
  \caption{Optional parameters for tasks (see~\S\protect\ref{sec:tasks}).}
  \label{tab:lqn-task-think-time}
\end{table}

\subsection{Entry Information}
\label{sec:lqn-entry-information}

Entries\dindex{LQN}{entry} are specified in the entry information section starting from
``\texttt{E}~\nonterminal{int}'' and ending with ``\texttt{-1}''.  The \nonterminal{int} parameter is either
the number of entry definitions in this section, or zero.  Each record in the entry section defines a single
parameter for an entry, such as its priority, or a single parameter for the phases of the entry, such as
service time.  Listing~\ref{lst:lqn-entry-info} shows the syntax for the most commonly used parameters.

\begin{lstlisting}[caption={Entry Information},label=lst:lqn-entry-info,frame=single,firstnumber=1,float]
E <int>
  A <entry-id> <activity-id>            # Start activity.
  F <entry-id> <entry-id> <real>        # forward.
  s <entry-id> <real> ... -1            # Service time by phase.
  y <entry-id> <entry-id> <real> ... -1 # Synchronous request by phase.
-1
\end{lstlisting}

All entry records start with a key letter, followed by an \nonterminal{entry-id}, followed by from one to up
to five arguments.  Table~\ref{tab:lqn-entry-specifier} lists all the possible entry specifiers.  The table
is split into six classes, based on the arguments to the parameter.  Records used to specifiy service time
and call rate parameters for phases take a list of from one to three arguments and terminated with a
`\texttt{-1}'.  All other entry records, with the exception of histogram information, take a fixed number of
arguments.   Records which only apply to the simulator are marked with a `\dag'.

\begin{table}
  \centering
  \begin{tabular}{|l|p{1.3in}|p{2.4in}|}
    \hline
    \multicolumn{1}{|c|}{\textbf{Key}}
    &\multicolumn{1}{c|}{\textbf{Paramater}}
    &\multicolumn{1}{c|}{\textbf{Arguments}}\\
    \hline
    \hline
    \multicolumn{3}{|c|}{\emph{One argument}}\\
    \hline
    \texttt{a} \nonterminal{entry-id} \nonterminal{real} & Arrival Rate   & \\
    \hline
    \texttt{A} \nonterminal{entry-id} \nonterminal{activity-id}& Start activity & \\
    \hline
    \texttt{p} \nonterminal{entry-id} \nonterminal{int} & Entry priority & \\
    \hline
    \hline
    \multicolumn{3}{|c|}{\emph{One to three phase arguments}}\\
    \hline
    \texttt{s} \nonterminal{entry-id} \nonterminal{real} $...$ -1 & Service Time. & The entry's \nonterminal{entry-id} and mean service
    time\index{entry!service time} value per phase.\\
    \hline
    \texttt{c} \nonterminal{entry-id} \nonterminal{real} $...$ -1 & Coefficient of Variation Squared.~\index{coefficient of variation} & The entry's
    \nonterminal{entry-id} and $\textit{CV}^2$ value for each phase.\\
    \hline
    \texttt{f} \nonterminal{entry-id} \nonterminal{int} $...$ -1 & Call Order & \texttt{STOCHASTIC} or \texttt{DETERMINISTIC} \\
    \hline
    \texttt{M} \nonterminal{entry-id} \nonterminal{real} $...$ -1 & Max Service Time\dag & Output probability that the service time result exceeds the
    \nonterminal{real} parameter, per phase.\\
    \hline
    \hline
    \multicolumn{3}{|c|}{\emph{Arguments for a single phase}}\\
    \hline
    \multicolumn{2}{|l|}{\texttt{H} \nonterminal{int} \nonterminal{entry-id} \nonterminal{real} :
      \nonterminal{real} \nonterminal{opt-int}}  & Histogram\dag:  An \nonterminal{int} phase, followed by a range from
    \nonterminal{real} to \nonterminal{real}, and an optional \nonterminal{int} buckets. \\
    \hline
    \hline
    \multicolumn{3}{|c|}{\emph{Destination and one argument}}\\
    \hline
    \texttt{F} \nonterminal{entry-id} \nonterminal{real} -1 & Forwarding Probability & Source and Destination entries, and probability reply is forwarded.\\
    \hline
    \hline
    \multicolumn{3}{|c|}{\emph{Destination and one to three phase arguments}}\\
    \hline
    \texttt{y} \nonterminal{entry-id} \nonterminal{entry-id} \nonterminal{real} $...$ -1 & Rendevous Rate & Source and Destination entries, and rate per phase.\\
    \hline
    \texttt{z} \nonterminal{entry-id} \nonterminal{entry-id} \nonterminal{real} $...$ -1 & Send-no-Reply Rate & Source and Destination entries, and rate per phase.\\
    \hline
    \hline
    \multicolumn{3}{|c|}{\emph{Semaphores and Locks}\dag} \\
    \hline
    \texttt{P}  \nonterminal{entry-id} & Signal\dag & Entry \nonterminal{entry-id} is used to \emph{signal}\index{signal} a semaphore
    task\index{semaphore task}.\\
    \hline
    \texttt{V}  \nonterminal{entry-id} & Wait\dag   & \\
    \hline
    \texttt{R}  \nonterminal{entry-id} & Read lock\dag & \\
    \hline
    \texttt{U}  \nonterminal{entry-id} & Read unlock\dag  & \\
    \hline
    \texttt{W}  \nonterminal{entry-id} & Write lock\dag   & \\
    \hline
    \texttt{X}  \nonterminal{entry-id} & Write unlock\dag & \\
    \hline
  \end{tabular}
  \caption{Entry Specifiers\index{LQN!entry}}
  \label{tab:lqn-entry-specifier}
\end{table}

\subsection{Activity Information}
\label{sec:lqn-activity-information}

Activity information\index{LQN!activity}\index{activity!LQN} sections are required to specify the parameters
and connectivity of the activities for a task.  Note that unlike all other sections, each task with
activities has its own activity information section. 

An activity information section starts with ``\texttt{A}~\nonterminal{task-id}'' and ends with
``\texttt{-1}''.  The data within an activity information section is partitioned into two parts.  The first
part lists the parameter data for an activity in a fashion similar to the parameter data for an entry; the
second section defines the connectivity of the activities.  Listing~\ref{lst:lqn-activity-info} shows the
basic syntax.

\lstset{language=LQN,basicstyle=\ttfamily,numbersep=10pt,firstnumber=1}
\begin{lstlisting}[caption={Activity Information},label=lst:lqn-activity-info,frame=single,firstnumber=1]
A <task-id>
  s <activity-id> <real>
  c <activity-id> <real>
  f <activity-id> <int>
  y <activity-id> <entry-id> <real>
  z <activity-id> <entry-id> <real>
:
  <activity-list> -> <activity-list>
-1
\end{lstlisting}

\begin{table}
  \centering
  \begin{tabular}{|l|l|p{3.5in}|}
    \hline
    \multicolumn{1}{|c|}{\textbf{Key}}
    &\multicolumn{1}{c|}{\textbf{Paramater}}
    &\multicolumn{1}{c|}{\textbf{Arguments}}\\
    \hline
    \multicolumn{3}{|c|}{\emph{one to three phase arguments}}\\
    \hline
    \texttt{s} & Service Time. & The entry's \nonterminal{entry-id} and mean service time value per phase.\\
    \hline
    \texttt{c} & Coefficient of Variation Squared.~\index{coefficient of variation} & The entry's
    \nonterminal{entry-id} and $\textit{CV}^2$ value for each phase.\\
    \hline
    \texttt{f} & Call Order & \texttt{STOCHASTIC} or \texttt{DETERMINISTIC} \\
    \hline
    \multicolumn{3}{|c|}{\emph{Destination and one to three phase arguments}}\\
    \hline
    \texttt{y} & Rendevous Rate & Source and Destination entries, and rate per phase.\\
    \hline
    \texttt{z} & Send-no-Reply Rate & Source and Destination entries, and rate per phase.\\
    \hline
  \end{tabular}
  \caption{Activity Specifiers\index{LQN!activity}}
  \label{tab:lqn-activity-specifier}
\end{table}

\begin{table}
  \centering
  \begin{tabular}{|l|p{3.5in}|}
    \hline
    \multicolumn{2}{|c|}{\emph{Post (or Join) lists}} \\
    \hline
    \nonterminal{activity-id} & \\
    \hline
    \nonterminal{activity-id} $+$ \nonterminal{activity-id} $+ ...$ & \\
    \hline
    \nonterminal{activity-id} $\&$ \nonterminal{activity-id} $\& ...$ & \\
    \hline
    \hline
    \multicolumn{2}{|c|}{\emph{Pre (or Fork) lists}} \\
    \hline
    \nonterminal{activity-id} & \\
    \hline
    \nonterminal{activity-id} $+$ \nonterminal{activity-id} $+ ...$ & \\
    \hline
    \nonterminal{activity-id} $\&$ \nonterminal{activity-id} $\& ...$ & \\
    \hline
  \end{tabular}
  \caption{Activity Lists\dindex{LQN}{activity list}}
  \label{tab:lqn-activity-lists}
\end{table}

\index{LQN|)}

\section{SPEX: Software Performance Experiment Driver}
\label{sec:spex}
\index{SPEX|(textbf}

SPEX, the \textbf{S}oftware \textbf{P}erformance \textbf{E}x\textbf{P}eriment driver, was originally a Perl
program used to generate and solve multiple layered queueing network models.  With version 5 of the solvers
this functionality has been incorporated into the \texttt{lqiolib}\index{lqiolib} and
\texttt{lqx}\index{lqx} libraries used by the simulator and analytic solver.  The primary benefit of this
change is that analytic solutions can run faster for reasons described
in~\cite{perf:mroz-2009-valuetools-lqx}.

SPEX augments the input file described in \S\ref{sec:lqn-file-format} by adding
\emph{variables}\dindex{spex}{variables} for setting input values, a \emph{Report Information}
(\S\ref{sec:lqn-report-information}) used to format output, and an optional \emph{Convergence Information}
(\ref{sec:lqn-convergence-information}) for feeding result values back into input variables.
Listing~\ref{lst:lqn-file-layout} shows these sections starting with comments in
\color{red}red\color{black}.  The syntax of these sections are described next in the order in which they
appear in the input model.

\subsection{Variables}
\label{sec:spex-variables}\index{SPEX!variables|(textbf}

SPEX variables are used to set and possibly vary various input values to the model, and to record results
from the solution of the model.  There are four types of variables:
control, scalar, array and observation.  Control variables are used to define parameters that control the
execution of the solver.  Scalar and array variables are used to set input parameters to the model.
Finally, observation variables are used to record results such as throughputs and utilizations.

\subsubsection{Control Variables}
\label{sec:spex-control}\index{SPEX!variables!control}

Control variables are used to set parameters that are used to control either the analytic solver
\emph{lqns}, or the simulator \emph{lqsim}.  With the exception of \ctrlparam{\$comment}, all of these
variables can be changed as SPEX executes, though this behaviour may not be appropriate in many cases.
Table~\ref{tab:spex-control} lists all of the control variables accepted by SPEX.  See Table~\ref{tab:spex-obsolete-control} for 
control variables that are no longer recognized.

\begin{table}
  \centering
  \begin{tabular}{|l|l|l|c|}
\hline
  \multicolumn{1}{|c|}{\textbf{Control Variable}}
  &\multicolumn{1}{c|}{\textbf{Type of Value}}
  &\multicolumn{1}{c|}{\textbf{Default Value}}
  & \textbf{Program}\\
  \hline
  \ctrlparam{\$model\_comment}  	    & \nonterminal{string}  & ""      &       \\
  \ctrlparam{\$solver}		            & \nonterminal{string}  & \multicolumn{2}{c|}{\emph{deprecated}} \\
  \hline
  \ctrlparam{\$convergence\_limit}	    & \nonterminal{real}    & 0.00001 & lqns  \\
  \ctrlparam{\$iteration\_limit}	    & \nonterminal{int}	    & 100     & lqns  \\
  \ctrlparam{\$print\_interval}		    & \nonterminal{int}	    & 1       & lqns  \\
  \ctrlparam{\$underrelaxation}             & \nonterminal{real}    & 0.9     & lqns  \\
  \hline
  \ctrlparam{\$block\_time}                 & \nonterminal{int}     & 50000   & lqsim \\
  \ctrlparam{\$number\_of\_blocks}          & \nonterminal{int}     & 1       & lqsim \\
  \ctrlparam{\$result\_precision}           & \nonterminal{real}    & --      & lqsim \\
  \ctrlparam{\$seed\_value}                 & \nonterminal{int}     & --      & lqsim \\
  \ctrlparam{\$warm\_up\_loops}             & \nonterminal{int}     & --      & lqsim \\
  \hline
  \ctrlparam{\$convergence\_iters}          & \nonterminal{int}     & --      & spex \\
  \ctrlparam{\$convergence\_under\_relax}   & \nonterminal{real}    & --      & spex \\
  \hline
  \end{tabular}
  \caption{Spex Control Variables}
  \label{tab:spex-control}
\end{table}

\subsubsection{Scalar Variables}
\label{sec:spex-scalar}\index{SPEX!variables!scalar}

Scalar variables are used to set input values for the model and are initialized using any
\nonterminal{ternary-expression} (\texttt{?:}\index{?:}\index{SPEX!ternary expressions}) using this syntax:
\begin{quotation}
    \$name = \nonterminal{ternary-expression}
\end{quotation}
The \nonterminal{ternary-expression} may contain any variables defined previously or subsequently in the
input file; order does not matter.  However, undefined variables and observation variables (more on these
below) are not permitted.  If \nonterminal{ternary-expression} is an actual ternary expression, the test
part must evaluate to a boolean.  Refer to Appendix~\ref{sec:old-grammar},~\S\ref{sec:spex-expressions} for
the complete grammar for \nonterminal{ternary-expression}.

\subsubsection{Array Variables}
\label{sec:spex-array}\index{SPEX!variables!array}

Array variables are used to specify a range of values that an input parameter may take on.  There are two
ways to specify this information:
\begin{lstlisting}[numbersep=10pt,firstnumber=1]
  $name = [x, y, z, ...]
  $name = [a : b, c]
\end{lstlisting}
The first form is used to set the variable \texttt{\$name} to the values in the list, \texttt{x},
\texttt{y}, \texttt{z}, \texttt{...}.  The second form is used the set the variable \texttt{\$name} from the
value \texttt{a} to \texttt{b} using a step size of \texttt{c}.  The value of \texttt{b} must be greater
that \texttt{a}, and the step size must be positive.  Regardless of the format, the values for array
variables must be constants.

During the execution of the solver, SPEX iterates over all of the values defined for each array variable. If
multiple arrays are defined, then SPEX generates the cross-product of all possible parameter values.  Note
that if a scalar variable is defined in terms of an array variable, then the scalar variable will be
recomputed for each model generated by SPEX.

\subsubsection{Observation Variables}
\label{sec:spex-observation}\index{SPEX!variables!observation}

There is a set of special symbols that are used to indicate to SPEX
which result values from the solution of the model are of interest.
This result indication has the following form:
\begin{quotation}
  \%\nonterminal{key}\nonterminal{phase} \$var
\end{quotation}
where \nonterminal{key} is a one or two letter key indicating the type of data to be observed and
\nonterminal{phase} is an optional integer indicating the phase of the data to be observed.  The data, once
obtained from the results of the model, is placed into the variable \$var where it may be used in the Result
Information section described below.

To obtain confidence interval information, the format is
\begin{quotation}
  \%\nonterminal{key}\nonterminal{phase}[confidence] \$var1 \$var2
\end{quotation}
where confidence can be 95 or 99, \$var1 is the mean and \$var2 is the half-width of the confidence interval.

The location of a result indication determines the entity to be observed.  Table~\ref{tab:obs-location}
describes each of the keys and where they may be used.

\begin{table}
  \centering
  \begin{tabular}{|l|l|l|l|l|}
    \hline
    \multicolumn{1}{|c|}{\textbf{Key}}
    &\multicolumn{1}{c|}{\textbf{Phase}}
    &\multicolumn{1}{c|}{\textbf{Description}}
    &\multicolumn{2}{l|}{\textbf{Location}}\\
\hline
  \texttt{\%u}  & no  & Utilization                    & processor declaration          & (p info) \\
                & yes &                                & task declaration               & (t info) \\
                & yes &                                & entry service declaration      & (s info) \\
\hline
  \texttt{\%f}  & no  & Throughput                     & task declaration               & (t info) \\
                & no  &                                & entry service declaration      & (s info) \\
\hline
  \texttt{\%pu} & no  & Processor Utilization          & task declaration               & (t info) \\
                & no  &                                & entry service declaration      & (s info) \\
\hline
  \texttt{\%s}  & yes & Service Time                   & entry service declaration      & (s info) \\
\hline
  \texttt{\%v}  & yes & Service Time Variance          & entry service declaration      & (s info) \\
\hline
  \texttt{\%fb} & no  & Throughput Bound               & entry service declaration      & (s info) \\
\hline
  \texttt{\%pw} & yes & Processor waiting time by task & entry service declaration      & (s info) \\
\hline
  \texttt{\%w}  & yes & Call waiting time              & entry call declaration         & (y info) \\
                & no  &                                & entry open arrival declaration & (a info) \\
\hline
  \end{tabular}
  \caption{Observation Key location}
  \label{tab:obs-location}
\end{table}

For any key/location combination that takes a phase argument, if none
is supplied then the sum of the values for all phases is reported.
This also happens if a phase of zero is given.

\label{sec:spex-variables|)}

\subsection{Report Information}
\label{sec:lqn-report-information}\index{SPEX!report}

The purpose of the report information section of the input file is to specify which variable values
(including result indications) are to be printed in the SPEX result file.  The format of this section is
shown in Listing~\ref{lst:lqn-report-info}.

\lstset{language=LQN,basicstyle=\ttfamily,numbersep=10pt,firstnumber=1}
\begin{lstlisting}[caption={Report Information},label=lst:lqn-report-info,frame=single,float]
R <int>
  $var
  $var = <ternary-expression>
-1
\end{lstlisting}

There may be any number of report declarations, however, the integer parameter to \texttt{R} must either be
the number of report declarations present or zero\footnote{The number is ignored; it is present in the
syntax so that the report section matches the other sections.}.

The \nonterminal{expression} may be any valid LQX expression as discussed above.  Note that in this section,
the report indication variables\index{variables!report indication} and the parameter
variables\index{variables!parameters} may both be used together.  The values of the variables listed in this
section are printed from left to right in the order that they appear in the input file separated by commas.
This output can then be used as input to Gnuplot\index{Gnuplot} or a spreadsheet such as Excel\index{Excel}.

There is a special variable called \texttt{\$0}\index{\$0} which represents the independent variable in the
results tables (the x-axis in plots).  The variable \texttt{\$0} may be set to any expression allowing for
flexibility in producing result tables.  This variable cannot be used as a parameter in the model.

\subsection{Convergence Information}
\label{sec:lqn-convergence-information}

Spex allows a parameter value to be modified at the end of a model solution and
then fed back in to the model.  The model is solved repeatedly until the parameter
value converges.  The convergence section is declared in a manner similar to the
result section:

\lstset{language=LQN,basicstyle=\ttfamily,numbersep=10pt,firstnumber=1}
\begin{lstlisting}[caption={Convergence Information},label=lst:lqn-convergence-info,frame=single]
C <int>
   $var
   $var = <ternary-expression>
-1
\end{lstlisting}

Convergence variables must be parameters.  They cannot be
result variables.

\subsection{Differeneces to SPEX 1}

This section outlines differences in the syntax between SPEX\index{SPEX!versions} version 1 and version 2.
SPEX version 1 was processed by a Perl\index{Perl!SPEX} program to convert the model into a conventional LQN
model file.  SPEX version 2 is now parsed directly and converted into LQX\index{SPEX!LQX} internally.

\subsubsection{Array Initialization}

Lists used for array initialization\index{SPEX!arrays} must now be enclosed within square brackets
(`[]')\index{[]}.  Further, the items must be separated using
commas. Figure~\ref{fig:spex-array-initialization} shows the old and new syntax.
\newsavebox{\spexone}
\newsavebox{\spextwo}
\begin{lrbox}{\spexone}
\begin{minipage}{.3\textwidth}
  \lstset{language=LQN,basicstyle=\ttfamily\color{red},numbersep=10pt,firstnumber=1}
\begin{lstlisting}
$array = 1 2 3
$array = 1:10,2
\end{lstlisting}
\end{minipage}
\end{lrbox}
\begin{lrbox}{\spextwo}
\begin{minipage}{.3\textwidth}
  \lstset{language=LQN,basicstyle=\ttfamily\color{blue},numbersep=10pt,firstnumber=1}
\begin{lstlisting}
$array = [1, 2, 3]
$array = [1:10,2]
\end{lstlisting}
\end{minipage}
\end{lrbox}
\begin{figure}
  \centering
  \subfloat[Spex 1]{\usebox{\spexone}}
  \subfloat[Spex 2]{\usebox{\spextwo}}
  \caption{x}
  \label{fig:spex-array-initialization}
\end{figure}

\subsubsection{Perl Expressions}

Perl Expressions are no longer supported in SPEX 2.0\index{SPEX!Perl}\index{Perl!SPEX}.  Rather, a subset of
LQX expressions are used instead.  For SPEX convergence
expressions\index{SPEX!convergence}\index{Convergence!SPEX}, Perl \texttt{if then else} statements must be
converted to use the ternary \texttt{?:}\index{?:}\index{SPEX!if-then-else}\index{SPEX!ternary expressions}
operator.  SPEX 2 cannot invoke Perl functions.

\subsubsection{Line Continuation}

Line continuation\index{line continuation}, where a line is terminated by a backslash (`\textbackslash')\index{\textbackslash}, is not needed
with Version 2 SPEX.  All whitespace, including newlines, is ignored.

\subsubsection{Comments}

In Version 1 of SPEX, all text before a dollar sign (`\$'), or either an upper case `G' or `P' at the start
of a line, was treated as a comment.  Since Version 2 SPEX is parsed directly, all comments must start with
the hash symbol (`\#')\index{\#}.

\subsubsection{String Substitution}

Version 2 SPEX does not support variable substitution of string parameters such as pragmas, and scheduling
types.  This restriction may be lifted in future versions.

\subsubsection{Pragmas}

Version 1 SPEX did not require the hash symbol (`\#')\index{\#} for setting pragmas.  Version 2 SPEX does.

\subsubsection{SPEX AGR}

SPEX AGR\index{SPEX!AGR} is no longer supported.

\subsubsection{Control Variables}
\label{sec:spex-obsolete-control}

Version 2 SPEX does not support the control variables shown in Table~\ref{tab:spex-obsolete-control}.
\begin{table}
  \centering
  \begin{tabular}{|l|}
\hline
  \multicolumn{1}{|c|}{\textbf{Control Variable}}\\
\hline
\ctrlparam{\$coefficient\_of\_variation}  \\
\ctrlparam{\$hosts}                       \\
\hline
  \end{tabular}
  \caption{Obsolete SPEX Control Variables.}
  \label{tab:spex-obsolete-control}
\end{table}

\subsubsection{Random Numbers}

Version 2 SPEX introduces the function \texttt{rand()}\index{rand@\texttt{rand()}}\index{SPEX!random
  numbers} to generate random numbers in the range of $[0..1)$.  To generate a set of experiments with
random parameters, an array (or set of arrays) with the number of elements corresponding to the number of
experiments is required to cause SPEX to iterate (see Section~\ref{sec:spex-and-lqx}).
Listing~\ref{lst:spex-rand} shows the syntax to generate random values for ten experiments.

\lstset{language=LQN,numbersep=10pt,firstnumber=1,texcl}
\begin{lstlisting}[float,caption={SPEX random parameter generation},label=lst:spex-rand]
  $experiments = [1:10,1]             # 10 experiments.
  $experiments, $parameter1 = rand()  # \$experiments is ignored.
  $experiments, $parameter2 = rand()  # \$experiments is ignored.
\end{lstlisting}

\index{SPEX|)}

\subsection{SPEX and LQX}
\label{sec:spex-and-lqx}

SPEX uses LQX\index{LQX} to generate individual model files.  All scalar paramaters are treated as globally
scoped variables in LQX and can be used to set parameters in the model.  If the assignement expression for a
scalar variable does not reference any array variables, it is set prior to the iteration of any loop.
Otherwise, the scalar variable is set during the execution of the innermost loop of the program.

Array variables are used to generate \texttt{foreach} loops in the LQX program.  The variable defining the
array is local (i.e.\ without the `\$') with the name of the SPEX parameter.  Each array variable generates
a for loop; the loops are nested in the order of the definition of the array variable.  The value variable
for the \texttt{foreach} loop is global (i.e.\ with the `\$') with the name of the SPEX parameter and can be
used as a parameter in the model.

If SPEX convergence is used, a final innermost loop is created.  This loop tests the variables defined in
the convergence section for change, and if any of the variables changes by a non-trivial amount, the loop
repeats.

Listing~\ref{lst:spex-file-layout} shows a model defined defined using SPEX syntax.
Listing~\ref{lst:spex-lqx} shows the corresponding LQX program generated by the model file.  

\lstset{language=LQN,basicstyle=\ttfamily,numbersep=10pt,firstnumber=1}
\begin{lstlisting}[float,caption={SPEX file layout.},label=lst:spex-file-layout]
$m_client = [1, 2, 3]
$m_server = [1: 3, 1]
$s_server = $m_server / 2

P 2
  p client i
  p server s 0.1
-1

T 2
  t client r client -1 client m $m_client %f $f_client
  t server n server -1 server m $m_server %u $u_server
-1

E 2
  s client 1 -1
  y client server $s_server -1
  s server 1 -1
-1

R 3
  $0
  $f_client
  $u_server
-1
\end{lstlisting}

\lstset{language=LQX,basicstyle=\ttfamily,numbersep=10pt,firstnumber=1,texcl=false}
\begin{lstlisting}[float,caption={LQX Program for SPEX input.},label=lst:spex-lqx]
m_client = array_create(1, 2, 3);
_0 = 0;
_f_client = 0;
_u_server = 0;
println_spaced(", ", "$0", "$f_client", "$u_server");
foreach( $m_client in m_client ) { 
  for ( $m_server = 1; $m_server <= 3; $m_server = ($m_server + 1)) {
    $s_server = ($m_server / 2);
    _0 = (_0 + 1);
    if (solve()) {
      _f_client = task("client").throughput;
      _u_server = task("server").utilization;
      println_spaced(", ", _0, _f_client, _u_server);
    } else {
      println("solver failed: $0=", _0);
    }
  }
}
\end{lstlisting}




%%% Local Variables:
%%% mode: latex
%%% mode: outline-minor
%%% fill-column: 108
%%% TeX-master: "userman"
%%% End:
