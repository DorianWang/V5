% 
% $Id: vector.tex 14472 2021-02-11 16:43:45Z greg $
% 
% ----------------------------------------------------------------------
\section{Vectors}
\label{sec:vector}

Vector objects have the following properties:\begin{itemize}
\item range checked.
\item indexed from \math[1 to n]{1...n} rather than from \math[0 to n
  - 1]{0...(n-1)}. 
\item expandable.
\end{itemize}
The \code{Vector} class is a template so
vectors of any type can be created.  

\subsection{Class Vector}

\subsubsection{Instance Variables}

\begin{verbatim}
private:
    unsigned sz;
    Type *ia;
\end{verbatim}

\begin{description}
\item[sz] \texonly{---} The size of the array \var{ia}.
\item[ia] \texonly{---} The array of elements of type \code{Type}.
\end{description}
\subsubsection{Constructors for Vector}
\label{sec:vector-constructors}

Objects of type \code{Vector} are created using the following
constructor:

\begin{verbatim}
Vector( unsigned size=VectorSize );
Vector( const Type *ar, unsigned size );
Vector( const Vector<Type>& iA );
\end{verbatim}

The first form creates a new vector with all \var{size} elements
initialized to \var{Nil}.  If the argument \var{size} is not supplied,
a vector of size \var{VectorSize} is created.  The second form of the
constructor creates a vector with \var{size} elements, and sets their
initial values from the conventional array \var{ar}.  The final form
of the constructor duplicates the argument \var{iA}.

The vector can be enlarged using the \code{grow()} call.

\subsubsection{Initialization}
\begin{description}

\item[init] \texonly{---} Vector initialization.\\
\code{void init( const Type* ar, unsigned size );}

Initialize the vector from the conventional array \var{ar}.  The
number of elements to copy is specified with \var{size}.
\end{description}

\subsubsection{Copying}
\label{sec:vector-copying}

The following set of functions and operators are used to: copy, change the
size of, or add to the vector.  

\begin{description}
\item[grow] \texonly{---} Enlarge vector.\\
\code{void grow( const unsigned amt );}

\code{grow()}, adds \var{amt} elements to the vector.  The new
elements are not initialized.

\item[Assignment] \texonly{---} Copy operator.\\
The operator ``\code{=}'' copies the vector.  The size of the receiver
will be set to the size of the sender.  

\item[Insertion] \texonly{---} Concatenation operator.\\
\code{Vector<Type>\& operator < < ( const Type\& element );}\\
\code{Vector<Type>\& operator < < ( const Vector<Type>\& vector );}

Append an object to the vector.  There are two forms:
\begin{itemize}
\item \code{Type\&} \var{element} --- insert a single
  \var{element} to the end of the vector.
\item \code{const Vector<Type>\&} \var{vector} --- concatenate a
  complete \var{vector} to the vector.
\end{itemize}

\item[Extraction] \texonly{---} Concatenation operator\\
\code{Vector<Type>\& operator>>( const Type\& element );}

Insert \var{element} at the front of the vector.  

\item[findOrAdd] \texonly{---} Find or add element in vector.\\
\code{unsigned findOrAdd( const Type\& element );}

Locate \var{element} in the vector.  If it is found, return the
corresponding index.  Otherwise, append \var{element} and return the
index. 
\end{description}

\subsubsection{Instance Variable Access}
\label{sec:vector-ivar}

\begin{description}

\item[Index] \texonly{---} Index operator.\\
\code{Type\& operator[](const unsigned ix);}\\
\code{Type operator[](const unsigned ix);}

Return the element at the index \var{ix}.  The index is range checked.
\end{description}

\subsubsection{Queries}
\label{sec:vector-queries}

\begin{description}

\item[size] \texonly{---} Vector size.\\
\code{unsigned size() const;}

Return the size of the vector.

\item[find] \texonly{---} Find element in vector.\\
\code{int find( const Type\& ) const;}

The \code{find()} function returns the index of the first object which
equates to \var{elem}.  \var{elem} must respond to the ``\code{==}''
operator.  If no matching objects are found, \code{find()} returns -1.
\end{description}

\subsubsection{Input and Output}
\label{sec:vector-io}

\begin{description}
\item[Extraction] \texonly{---} Printing operator.\\
\code{ostream\& operator < < ( ostream\& os, const Vector<Type>\&
  vector);}\\
\code{void print( ostream\& os = cout ) const;}

The output operator \code{< <} writes the contents of \var{vector} to
the stream \var{os} in the form \kbd{<n>(1,2,...,n)}.  The member
function \code{print()} performs the same function.  If \var{os} is
not supplied, output is sent to \var{cout}.
\end{description}

\subsection{Class VectorIterator}
\label{sec:vector-iterator}


\subsubsection{Instance Variables}

\begin{verbatim}
private:
    const Vector<Type>& vecPtr;
    unsigned index;
\end{verbatim}

\begin{description}
\item[vecPtr] \texonly{---} A pointer to the vector that this iterator
  applies to.
\item[index] \texonly{---} The current index into the vector.
\end{description}


\subsubsection{Constructors for Class VectorIterator}

Vector Iterators are created by passing the associated
\link{vector}{sec:vector} to the \code{VectorIterator} constructor.
The name of the iterator object is used as a function to sequence
through the vector.  This class can only be used with vectors of
simple types (such as integer), or to vectors of pointers to objects.  

\begin{verbatim}
VectorIteratorPtr( const Vector<Type>& aVec );
\end{verbatim}

\subsubsection{Instance Methods}

\begin{description}
\item[()] \texonly{---} Iterator.\\
\code{Type operator()();}

Return the next object in the vector.  If there are no further
objects, return \bf{Nil} and reset \var{index} to zero.  The next
call to this function will then return the first object in the vector.

\end{description}

\subsection{Class VectorIteratorPtr}

This class is similar to the preceding one except that the function
operator returns \bf{pointers} to the elements in the vector rather
than the elements themselves.  It can be used for both simple and
complex types.

% Local Variables: 
% mode: latex
% TeX-master: "libmva"
% End: 
