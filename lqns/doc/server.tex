\C 
\C $Id: server.tex 5439 2005-01-02 02:30:31Z greg $
\C 
\C $Log$
\C Revision 31.0  2005/01/02 02:30:29  greg
\C Bring all files up to revision 31.
\C
\C Revision 30.0  2003/11/12 18:56:52  greg
\C Bring all files up to revision 30.0
\C
\C Revision 29.0  2003/06/13 00:48:38  greg
\C bounce version number.
\C
\C Revision 1.5  1995/01/11 21:05:12  greg
\C Revised for R12 of MVA software.
\C
\C Revision 1.4  1994/12/09  02:02:36  greg
\C update.
\C
\C Revision 1.3  1994/08/04  20:52:11  greg
\C Intermediate save
\C
\C Revision 1.2  1994/06/24  19:11:32  greg
\C Format changes.
\C
\C Revision 1.1  1994/06/23  19:11:01  greg
\C Initial revision
\C
\C ----------------------------------------------------------------------
\section{Server}
\label{sec:server}

The Server class defines {\em stations\/} in the queueing network
model.  


\subsection{Server Class Hierarchy}
\label{sec:server-hierarchy}

The class relationships for the MVA server module are shown in
\htmlonly{the \link{figure
    below}{fig:server}}\texonly{Figure~\ref{fig:server}}.  The class
\file{Server} is an abstract superclasses which is not used directly.

\begin{figure}[htbp]
  \label{fig:server}
  \begin{center}
    \T \tex \leavevmode \C 
\C $Id$
\C 
\C $Log$
\C Revision 31.0  2005/01/02 02:30:29  greg
\C Bring all files up to revision 31.
\C
\C Revision 30.0  2003/11/12 18:56:52  greg
\C Bring all files up to revision 30.0
\C
\C Revision 29.0  2003/06/13 00:48:38  greg
\C bounce version number.
\C
\C Revision 1.5  1995/01/11 21:05:12  greg
\C Revised for R12 of MVA software.
\C
\C Revision 1.4  1994/12/09  02:02:36  greg
\C update.
\C
\C Revision 1.3  1994/08/04  20:52:11  greg
\C Intermediate save
\C
\C Revision 1.2  1994/06/24  19:11:32  greg
\C Format changes.
\C
\C Revision 1.1  1994/06/23  19:11:01  greg
\C Initial revision
\C
\C ----------------------------------------------------------------------
\section{Server}
\label{sec:server}

The Server class defines {\em stations\/} in the queueing network
model.  


\subsection{Server Class Hierarchy}
\label{sec:server-hierarchy}

The class relationships for the MVA server module are shown in
\htmlonly{the \link{figure
    below}{fig:server}}\texonly{Figure~\ref{fig:server}}.  The class
\file{Server} is an abstract superclasses which is not used directly.

\begin{figure}[htbp]
  \label{fig:server}
  \begin{center}
    \T \tex \leavevmode \C 
\C $Id$
\C 
\C $Log$
\C Revision 31.0  2005/01/02 02:30:29  greg
\C Bring all files up to revision 31.
\C
\C Revision 30.0  2003/11/12 18:56:52  greg
\C Bring all files up to revision 30.0
\C
\C Revision 29.0  2003/06/13 00:48:38  greg
\C bounce version number.
\C
\C Revision 1.5  1995/01/11 21:05:12  greg
\C Revised for R12 of MVA software.
\C
\C Revision 1.4  1994/12/09  02:02:36  greg
\C update.
\C
\C Revision 1.3  1994/08/04  20:52:11  greg
\C Intermediate save
\C
\C Revision 1.2  1994/06/24  19:11:32  greg
\C Format changes.
\C
\C Revision 1.1  1994/06/23  19:11:01  greg
\C Initial revision
\C
\C ----------------------------------------------------------------------
\section{Server}
\label{sec:server}

The Server class defines {\em stations\/} in the queueing network
model.  


\subsection{Server Class Hierarchy}
\label{sec:server-hierarchy}

The class relationships for the MVA server module are shown in
\htmlonly{the \link{figure
    below}{fig:server}}\texonly{Figure~\ref{fig:server}}.  The class
\file{Server} is an abstract superclasses which is not used directly.

\begin{figure}[htbp]
  \label{fig:server}
  \begin{center}
    \T \tex \leavevmode \C 
\C $Id$
\C 
\C $Log$
\C Revision 31.0  2005/01/02 02:30:29  greg
\C Bring all files up to revision 31.
\C
\C Revision 30.0  2003/11/12 18:56:52  greg
\C Bring all files up to revision 30.0
\C
\C Revision 29.0  2003/06/13 00:48:38  greg
\C bounce version number.
\C
\C Revision 1.5  1995/01/11 21:05:12  greg
\C Revised for R12 of MVA software.
\C
\C Revision 1.4  1994/12/09  02:02:36  greg
\C update.
\C
\C Revision 1.3  1994/08/04  20:52:11  greg
\C Intermediate save
\C
\C Revision 1.2  1994/06/24  19:11:32  greg
\C Format changes.
\C
\C Revision 1.1  1994/06/23  19:11:01  greg
\C Initial revision
\C
\C ----------------------------------------------------------------------
\section{Server}
\label{sec:server}

The Server class defines {\em stations\/} in the queueing network
model.  


\subsection{Server Class Hierarchy}
\label{sec:server-hierarchy}

The class relationships for the MVA server module are shown in
\htmlonly{the \link{figure
    below}{fig:server}}\texonly{Figure~\ref{fig:server}}.  The class
\file{Server} is an abstract superclasses which is not used directly.

\begin{figure}[htbp]
  \label{fig:server}
  \begin{center}
    \T \tex \leavevmode \input{server.epic}
    \caption{Server class hierarchy}
    \H \htmlimage{server.gif}
  \end{center}
\end{figure}

\htmlrule
\C
\C ----------------------------------------------------------------------
\C
\subsection{Class Server  --- Abstract Superclass}
\subsubsection{Instance variables}
\label{sec:server-ivars}

\begin{verbatim}
public:
double *W;                  /* Waiting time per visit.      */
unsigned index;             /* Not used locally.            */
Interlock * IL;             /* Interlock stuff.             */

protected:             
const unsigned E;           /* Number of entries.           */
const unsigned K;           /* Number of Classes.           */
double ***s;                /* Service Time per phase.      */
double ***v;                /* Visit ratios per phase.      */
\end{verbatim}

\begin{description}
\item[E] \texonly{---} Number of \underline{E}ntries.
\item[IL] \texonly{---} Interlocked throughput.  Dimensioned by \bold{class}.
\item[K] \texonly{---} Number of Classes.
\item[s] \texonly{---} \underline{S}ervice time per visit.  Dimensioned by
  \bold{entry}, \bold{class} and \bold{phase}.
\item[v] \texonly{---} \underline{V}isit ratios. Dimensioned by
  \bold{entry}, \bold{class} and \bold{phase}.
\item[W] \texonly{---} \underline{W}aiting time per visit.  Dimensioned by \bold{class}.
\end{description}

\subsubsection{Constructors for Class Server}

Servers are created by calling the appropriate server type constructor
with two arguments, \var{e} and \var{k}.  The [optional] \var{e}
argument specifies the maximum number of entries for the server.  The
[optional] \var{k} argument specifies the number of classes.  If the
number of entries or classes is not specified, one is assumed.  These
values cannot be changed once the server is created.

\begin{verbatim}
Server();
Server( const unsigned k );
Server( const unsigned e, const unsigned k );
\end{verbatim}


\subsubsection{Initialization}
\label{sec:server-initialization}

\begin{description}
\item[initialize] \texonly{---} Initialization.\\
  \code{void initialize( const unsigned e, const unsigned k );}

  \code{Initialize()} is used to allocate storage for most of the arrays
  defined in this class.  The exceptions are the
  \link{populations}{sec:population} \var{L} and \var{U}; these instance
  variabels are allocated by the appropriate \link{MVA}{sec:mva} solver.

  \label{sec:server-initstep}
\item[initStep] \texonly{---} Per-MVA step initialization.\\
  \code{void initStep( const MVA& solver);}

  This method is called before the \link{MVA step}{sec:mva-step} for a
  given population.  Some server types need initialization while others
  do not.  By default, no operation is performed but is overridden by
  subclasses.

\item[check] \texonly{---} Simple model check.\\
  \code{void check();}

  Perform a simple model consistency check.
\end{description}

\subsubsection{Instance Variable Access}

Two forms exist for most member functions that access instance
variables.  The first takes two arguments: the \bold{class}, \var{k} and the
value being stored.  The second form includes the \bold{entry}, \var{e} and
\bold{phase}, \var{p}. 

\begin{description}
\item[setService] \texonly{---} Set service time.\\
  \code{Server& setService( const unsigned k, const double s);}\\
  \code{Server& setService( const unsigned e, const unsigned k, const
    unsigned p, const double s);}

  Set the service time for the station to \var{s}.  

\item[setVisits] \texonly{---} Set visit ratios.\\
  \code{Server& setVisits( const unsigned k, const double v );}\\
  \code{Server& setVisits( const unsigned e, const unsigned k, const
    double v );}

  Set the visit ratio for the station to \var{v}.

\item[addVisits] \texonly{---} Add to visit ratio.\\
  \code{Server& setVisits( const unsigned e, const unsigned k, const
    double v );}

  Add \var{v} to the present value of the visit ratio for the station.

\item[clearVisits] \texonly{---} Clear visit ratios.\\
  \code{clearVistis( const unsigned k );}

  Clear visit ratios for class \var{k}.

\item[setVariance] \texonly{---} Set variance.\\
  \code{Server& setVariance( const unsigned e, const unsigned k, const
    unsigned p, const double v);}\\
  \code{Server& setServerVariance( const double v);}

  Set the variance for the entry (or station) to \var{v}.  The server
  class must be one that uses variance in its waiting time calculation.
  Otherwise, a call to these functions will generate run-time errors.
\end{description}

\subsubsection{Queries}
\begin{description}
\item[nEntries] \texonly{---} Number of Entries.\\
  \code{unsigned nEntries() const;}

  Return the number of entries for this station.

\item[S] \texonly{---} Service Time.\\
  \code{double S( const unsigned e, const unsigned k ) const;}\\
  \code{double S( const unsigned e, const unsigned k, const unsigned p ) const;}\\
  \code{double etaS( const unsigned e, const unsigned k ) const;}\\

  Return the service time.  The first form of \code{S()} returns the
  total service time over all phases.  The second form returns the
  service time for the phase \var{p}.  The final form, \code{etaS()}
  returns the service time ratioed by the number of visits (see
  \link{\code{eta()}}{sec:server-eta}).

\item[V] \texonly{---} Visit ratios.\\
  \code{double V( const unsigned e, const unsigned k ) const;}\\
  \code{double V( const unsigned k ) const;}

  Return the visit ratios by [optionally] \bold{entry} and \bold{class}.

\item[R] \texonly{---} Residence time.\\
  \code{double R( const unsigned e, const unsigned k ) const;}\\
  \code{double R( const unsigned k ) const;}\\
  \code{double R() const;}

  Return the residence time given by \math[visits times waiting]{v_{ek}
    W_{ek}}.  Residence time can be returned by \bold{entry} and
  \bold{class}.  

\item[computeMarginalQueueSize] \texonly{---} Return true if necessary\\
  \code{int computeMarginalQueueSize();}

  Return non-zero value if the marginal queue sizes need to be
  calculated for the station.  By default, marginal queue sizes are NOT
  calclated. 

  \label{sec:server-eta}
\item[eta] \texonly{---} Visit ratio fraction.\\
  \code{double eta( const unsigned e, const unsigned k ) const;}

  Return the fraction of visits by \var{e} and \var{k} to the total
  number of visits to the station.  Private member function.

\item[prOt] \texonly{---} Overtaking probability.\\
  \code{double prOt( const unsigned e, const unsigned k, const unsigned
    p) const;}

  Return the overtaking probability.  The overtaking probabilty for most
  server types is zero by default.  Overridden by subclasses that
  support phases.

\end{description}

\subsubsection{Computation}

\begin{description}
\item[wait] \texonly{---} Find waiting time. \\
  \code{double wait( const MVA& solver, const unsigned e, const unsigned k, const
    PopVector & N ) const;}

  Calculate and return the waiting time using the \link{MVA}{sec:mva}
  solver, \var{solver} for \bold{entry} \var{e} and \bold{class} \var{k}
  at a \link{population vector}{sec:vector} \var{N}.  Implemented by subclasses.

\item[interlock] \texonly{---} Find interlocked throughput.\\
  \code{double interlock( const double lambda, const unsigned k) const;}

  Return the interlocked throughput at this station for a \bold{class}
  \var{k} for a given input throughput of \var{lambda}.  This function
  is used by the MVA sovler to adjust for interlocked flows.

\end{description}

\subsubsection{Input and Output}
\label{sec:server-print}

The following functions are used to print information about the
receiver.  \var{Ostream} specifies the output stream.

\begin{description}
\item[print] \texonly{---} Print input parameters.\\
  \code{ostream& print( ostream& os ) const;}

  Print input information about this station.  \code{Print()} calls
  \code{printInput()} for all entries and classes.

\item[printExtraInfo] \texonly{---} Print bonus infor\\
  \code{ostream& printExtraInfo( ostream& output, const unsigned = 0 );}

  Print out any extra information (such as overtaking probabilities) for
  the station.  By default, there is no extra information to print.

\item[printInput] \texonly{---} Print input parameters.\\
  \code{ostream& printInput( ostream&, const unsigned, const unsigned )
    const;}

  Print out the input values for Service Time
  \link{\var{s}}{sec:server-ivars} and Visit Ratios
  \link{\var{v}}{sec:server-ivars} for \bold{entry} \var{e},
  \bold{class} \var{k}. 

\item[printResults] \texonly{---} Print results.\\
  \code{ostream& printResults( ostream& os ) const;}

  Print result information about this station.  \code{Print()} calls
  \code{printEntry()} for all entries and classes.

\item[printEntry] \texonly{---} Print results.\\
  \code{ostream& printEntry( ostream& os, const unsigned e, const
    unsigned k ) const;}

  Print Queue Length \link{\var{L}}{sec:server-ivars}, Waiting Time
  \link{\var{W}}{sec:server-ivars}, throughput and Utilization
  \link{\var{U}}{sec:server-ivars} for this station for \bold{entry}
  \var{e} and \bold{class} \var{k}.
\end{description}

\subsection{Class FCFS_Server}
\label{sec:server-fcfs}

\subsection{Class Delay_Server}
\label{sec:server-delay}

\subsection{Class HVFCFS_Server}
\label{sec:server-hvfcfs}

\subsection{Class Phased_Server}
\label{sec:server-phased}

\subsection{Class HVFCFS_Phased_Server}
\label{sec:server-hvfcfs-phased}

\subsection{Class Blended_HVFCFS_Server}
\label{sec:server-blended-hvfcfs}

\subsection{Class Blended_HVFCFS_Phased_Server}
\label{sec:server-blended-hvfcfs-phased}

\C
\C ----------------------------------------------------------------------
\C
\subsection{Class Simple_Multi_Server}
\label{sec:simple-multiserver}

This class implements \link{BCMP-type (or product form)
  multiservers}[~\Cite]{queue:reiser-80}.  As with simple First-Come,
First Served servers, the service time for each class at a given
server should be the same.  

\subsubsection{Instance Variables}
\label{sec:simple-multiserver-ivars}
\begin{verbatim}
unsigned J;                 /* Number of servers.           */
\end{verbatim}

\begin{description}
\item[M] \texonly{---} Number of stations.
\end{description}

\subsubsection{Constructors for Simple_Multi_Server}

Constructors for multiple server stations take an additional argument,
\var{copies}, that sets the number of servers for the station object.
The number of classes \var{k} and the number of entries \var{e} are
both optional.

\begin{verbatim}
    Simple_Multi_Server( const unsigned copies );
    Simple_Multi_Server( const unsigned k, const unsigned copies );
    Simple_Multi_Server( const unsigned e, const unsigned k, const unsigned copies );
\end{verbatim}

\subsubsection{Initialization}

\begin{description}
\item[initialize] \texonly{---} Initialization\\
  \code{void initialize();}

  The initialization function merely checks that the user defined at
  least one server.
\end{description}

\subsubsection{Instance Variable Access}

\begin{description}
\item[marginalProbabilitiesSize()] \texonly{---} Return the number of servers.\\
  \code{int marginalProbabilitiesSize() const;}

  Return the number of servers \math[\var{J}]{J}.

\end{description}

\subsubsection{Computation}
\begin{description}

\item[wait] \texonly{---} Waiting Time\\
  \code{double wait( const MVA&, const unsigned, const unsigned, const PopVector & ) const;}

  The following expression, from \link{Reiser and
    Lavenberg}[~\Cite]{queue:reiser-80}, equation (3.8), is used to find
  the waiting time
  \begin{iftex}
    \tex
    \begin{equation}
      W_{mk}({\bf N}) = \frac{ s_{mk} }{J_{m}} \left[ 1 + L_{mk}({\bf N} -
      e_k) + \sum_{j=0}^{J_{m} - 2} (J_{m} - 1 - j) P(j,{\bf N} - e_k) \right]
    \end{equation}
  \end{iftex}

\end{description}

\subsubsection{Input and Output}
\begin{description}
\item[print] \texonly{---} Print input data\\
  \code{ostream& print( ostream& output ) const;}

  Print information about this station.
\end{description}

\C
\C ----------------------------------------------------------------------
\C
\subsection{Class Multi_Server}
\label{sec:multiserver}

\subsubsection{Constructors for Multi_Server}

\begin{verbatim}

\end{verbatim}

\subsubsection{Computation}
\begin{description}

\item[wait] \texonly{---} Waiting Time\\
  \code{double wait( const MVA&, const unsigned, const unsigned, const PopVector & ) const;}

  The waiting time for multiservers with service times that differ for
  each chains is from \link{Eqn (19)}[~\Cite]{queue:deSouzaeSilva-87}
  (also: \link{Eqn (2.7)}[~\Cite]{queue:conway-88}).

  \begin{iftex}
    \tex
    \begin{equation}
      W_{mk}({\bf N}) = s_{mk}
      + \sum_{i=1}^{K}{\it XE}_{mki}({\bf N}) L_{mi}^{*}({\bf N} - e_k)
      + {\it XR}_{mk}({\bf N})P_{m}(J_{m},{\bf N} - e_k) 
    \end{equation}
  \end{iftex}
  
  \begin{iftex}
      \tex
    \begin{eqnarray}
      {\it XE}_{mki}({\bf N}) & = & \sum_{{\bf n} \in A_{ik}}\frac{{\it
          PS}_{mk}({\bf n},i,{\bf N})}{n_1\mu_{m1}+...+n_K\mu_{mK}}
      \label{eqn:msfcfs-xe} \\
      {\it XR}_{mk}({\bf N}) & = & \sum_{{\bf n} \in B_{k}}\frac{{\it
          PS}_{mk}({\bf n},{\bf N})}{n_1\mu_{m1}+...+n_K\mu_{mK}}
      \label{eqn:msfcfs-xr} \\
      {\it PS}_{mk}({\bf n},i,{\bf N}) & = & \frac{A_m( {\bf n}, {\bf N} -
        e_k )}{C_{mi}( {\bf N} - e_k )}
      \\
      {\it PS}_{mk}({\bf n},{\bf N}) & = & \frac{A_m( {\bf n}, {\bf N} -
        e_k )}{C_m( {\bf N} - e_k )}
      \\
      A_m({\bf n},{\bf N}-e_k) & = & \frac{J_{m}!}{n_1!n_2!...n_K!} \left[
      \prod_{x=1}^{K} F_{mx}^{n_{x}}({\bf N}-e_k) \right]
      \\
      C_{mi}({\bf N}-e_k) & = & \sum_{{\bf n} \in A_{ik}}A_m({\bf n},{\bf N}-e_k)
      \\
      C_m({\bf N}-e_k) & = & \sum_{{\bf n} \in B_{k}}A_m({\bf n},{\bf
        N}-e_k)
      \\ 
      F_{mx}({\bf N}-e_k) & = & \frac{U_{mx}({\bf
          N}-e_k)}{\sum_{y=1}^{K} U_{my}({\bf N}-e_k)}
      \label{eqn:msfcfs-f} \\
      L_{mi}^{*}({\bf N} - e_k) & = & L_{mi}({\bf N} - e_k) -
      U_{mi}({\bf N} - e_k)
    \end{eqnarray}
  \end{iftex}

  The populations for the vector {\bf n} used in the summations of the
  \var{XE} and \var{XR} terms in the waiting time expression,
  (\link{Eqns (16) and (17)}[~\Cite]{queue:deSouzaeSilva-87}), are
  generated by objects of the classes
  \link{\file{A_Iterator}}[~(\Sec\Ref)]{sec:pop-AIter} and
  \link{\file{B_iterator}}[~(\Sec\Ref)]{sec:pop-BIter} respectively.
  \begin{iftex}
    \tex
    The populations satisfy the constraints of (eqn:pop-A) and
    (eqn-pop-B). 
    \begin{eqnarray}
      A_{ik} & = & \leftbracket {\bf n} | n_1+...+n_K = J_{m}, n_j \geq 0, j =
      1,...,K, {\bf n} \leq ({\bf N} - e_k), n_i \geq 1 \rightbracket
      \label{eqn:pop-A}\\ 
      B_{k} & = & \leftbracket {\bf n} | n_1+...+n_K = J_{m}, n_j \geq 0, j =
      1,...,K, {\bf n} \leq ({\bf N} - e_k) \rightbracket
      \label{eqn:pop-B}
    \end{eqnarray}
  \end{iftex}

\item[effectiveBacklog] \texonly{---} XE term.\\
  \code{double effectiveBacklog( const MVA& solver, const PopVector& N,
    const unsigned k ) const;}

  The mean effective backlog of work in the queue when a chain \var{k}
  customer arrives, \math[\var{XE}]{XE_{mk}} (\link{Eqn
    (18)}[~\Cite]{queue:deSouzaeSilva-87}), multiplied by the number
  of customers in the queue.

  \begin{iftex}
    \tex
    \begin{displaymath}
      \sum_{i=1}^{K}{\it XE}_{mki}({\bf N}) L_{mi}^{*}({\bf N} - e_k)
    \end{displaymath}
  \end{iftex}

\item[departureTime] \texonly{---} XR term.\\
  \code{double departureTime( const MVA& solver, const PopVector& N, const unsigned k ) const;}

  The average elapsed time until the first customer leaves,
  \math[\var{XR}]{XR_{mk}} \link{(Eqn
    (16))}[~\Cite]{queue:deSouzaeSilva-87}, multiplied by the
  probability that all servers are busy,
  \math[\var{P(J,N-ek)}]{P_{m}(J_{m},{\bf N} - e_k)}.

  \begin{iftex}
    \tex
    \begin{displaymath}
      {\it XR}_{mk}({\bf N})P_{m}(J_{m},{\bf N} - e_k) 
    \end{displaymath}
  \end{iftex}

\item[sumOf_PS_k] \texonly{---} \math[Sum of A(n,N-ek)/C(n,N-ek)]{\sum
    \frac{A_m( {\bf n}, {\bf N} - e_k )}{C_m( {\bf N} - e_k )}}\\ 
  \code{double sumOf_PS_k( const MVA& solver, const PopVector& N,
    const unsigned k, PopulationIterator& next ) const;}

  Common expression to the XE and XR calculation \link{(Eqns (9) and
    (13))}[~\Cite]{queue:deSouzaeSilva-87}.  The `\var{next}' argument
  determines the values that are used in the sum.

  \begin{iftex}
    Equations~(\ref{eqn:msfcfs-xe}) and (\ref{eqn:msfcfs-xr}) are
    re-expressed as:

    \tex
    \begin{eqnarray*}
      {\it XE}_{mki}({\bf N}) & = & \frac{\displaystyle \sum_{ {\bf n} \in
        A_{ik}} \frac{ A_{m}({\bf n},{\bf N} - e_k) }{
        n_1\mu_{m1}+...+n_K\mu_{mK} } }{\displaystyle \sum_{{\bf n} \in A_{ik}}
      A_{m}({\bf n},{\bf N} - e_k) }
      \label{eqn:msfcfs-xe-new} \\
      {\it XR}_{mk}({\bf N}) & = & \frac{\displaystyle \sum_{ {\bf n} \in
        B_{k}} \frac{A_{m}({\bf n},{\bf N} - e_k) }{
        n_1\mu_{m1}+...+n_K\mu_{mK} } }{\displaystyle \sum_{{\bf n} \in B_{k}}
      A_{m}({\bf n},{\bf N} - e_k) }
      \label{eqn:msfcfs-xr-new}
    \end{eqnarray*}
  \end{iftex}

\item[A] \texonly{---} \math[\var{A}]{A_m} term \\
  \code{const MVA& solver, const PopVector& n, const PopVector& N, const unsigned k ) const;}

  The \math[\var{A}]{A_m} term \link{(Eqn
    10)}[~\Cite]{queue:deSouzaeSilva-87}.

  \begin{iftex}
    \tex
    \begin{displaymath}
      A_m({\bf n},{\bf N}-e_k) = J_{m}! \cdot \frac{\displaystyle
      \prod_{x=1}^{K} F_{mx}^{n_{x}}({\bf N}-e_k) }{\displaystyle
      \prod_{x=1}^{K} n_{x}! }
    \end{displaymath}
  \end{iftex}

\item[F] \texonly{---} Fraction of utilization \\
  \code{double F( const MVA& solver, const unsigned i, const PopVector& N, const unsigned k ) const;}

  The fraction of class \var{i} utilization at the server \link{(Eqn
  (11))}[~\Cite]{queue:deSouzaeSilva-87}).

  \begin{iftex}
    \tex
    \begin{displaymath}
      F_{mx}({\bf N}-e_k) = \frac{ U_{mx}({\bf N}-e_k) }{
        \sum_{y=1}^{K} U_{my}({\bf N}-e_k) }
    \end{displaymath}
  \end{iftex}

\item[meanMinimumService] \texonly{---} Mean minimum service time\\
  \code{double meanMinimumService( const PopVector& N ) const;}

  The elapsed time for the arrival of this customer until the next
  departure \link{Eqn (15)}[~\Cite]{queue:deSouzaeSilva-87}.  It is
  simply the minimum of the service times of customers in service for
  exponentially distributed service times:
  
  \begin{iftex}
    \tex
    \begin{displaymath}
      \frac{ 1 }{ \sum_{y=1}^{K} \frac{ n_{k} }{ s_{k} } }
    \end{displaymath}
  \end{iftex}

\end{description}


\subsubsection{Input and Output}

\begin{description}
\item[print] \texonly{---} Print input data\\
  \code{ostream& print( ostream& output ) const;}

  Print information about this station.
\end{description}


\subsection{Class MOL Multiserver}
\label{sec:mol-multiserver}

\subsubsection{Instance Variables}
\label{sec:mol-multiserver-ivars}
\begin{verbatim}

\end{verbatim}

\begin{description}
\item[wait] \texonly{---} Waiting Time\\
  \code{double wait( const MVA&, const unsigned, const unsigned, const
    PopVector & ) const;}

  The waiting time expression for...

\end{description}

\subsubsection{Constructors for MOL Multiserver}

\begin{verbatim}

\end{verbatim}

\subsubsection{Initialization}
\subsubsection{Copying}
\subsubsection{Instance Variable Access}
\subsubsection{Queries}
\subsubsection{Computation}
\subsubsection{Input and Output}


\C Local Variables: 
\C mode: latex
\C TeX-master: "lqns"
\C End: 

    \caption{Server class hierarchy}
    \H \htmlimage{server.gif}
  \end{center}
\end{figure}

\htmlrule
\C
\C ----------------------------------------------------------------------
\C
\subsection{Class Server  --- Abstract Superclass}
\subsubsection{Instance variables}
\label{sec:server-ivars}

\begin{verbatim}
public:
double *W;                  /* Waiting time per visit.      */
unsigned index;             /* Not used locally.            */
Interlock * IL;             /* Interlock stuff.             */

protected:             
const unsigned E;           /* Number of entries.           */
const unsigned K;           /* Number of Classes.           */
double ***s;                /* Service Time per phase.      */
double ***v;                /* Visit ratios per phase.      */
\end{verbatim}

\begin{description}
\item[E] \texonly{---} Number of \underline{E}ntries.
\item[IL] \texonly{---} Interlocked throughput.  Dimensioned by \bold{class}.
\item[K] \texonly{---} Number of Classes.
\item[s] \texonly{---} \underline{S}ervice time per visit.  Dimensioned by
  \bold{entry}, \bold{class} and \bold{phase}.
\item[v] \texonly{---} \underline{V}isit ratios. Dimensioned by
  \bold{entry}, \bold{class} and \bold{phase}.
\item[W] \texonly{---} \underline{W}aiting time per visit.  Dimensioned by \bold{class}.
\end{description}

\subsubsection{Constructors for Class Server}

Servers are created by calling the appropriate server type constructor
with two arguments, \var{e} and \var{k}.  The [optional] \var{e}
argument specifies the maximum number of entries for the server.  The
[optional] \var{k} argument specifies the number of classes.  If the
number of entries or classes is not specified, one is assumed.  These
values cannot be changed once the server is created.

\begin{verbatim}
Server();
Server( const unsigned k );
Server( const unsigned e, const unsigned k );
\end{verbatim}


\subsubsection{Initialization}
\label{sec:server-initialization}

\begin{description}
\item[initialize] \texonly{---} Initialization.\\
  \code{void initialize( const unsigned e, const unsigned k );}

  \code{Initialize()} is used to allocate storage for most of the arrays
  defined in this class.  The exceptions are the
  \link{populations}{sec:population} \var{L} and \var{U}; these instance
  variabels are allocated by the appropriate \link{MVA}{sec:mva} solver.

  \label{sec:server-initstep}
\item[initStep] \texonly{---} Per-MVA step initialization.\\
  \code{void initStep( const MVA& solver);}

  This method is called before the \link{MVA step}{sec:mva-step} for a
  given population.  Some server types need initialization while others
  do not.  By default, no operation is performed but is overridden by
  subclasses.

\item[check] \texonly{---} Simple model check.\\
  \code{void check();}

  Perform a simple model consistency check.
\end{description}

\subsubsection{Instance Variable Access}

Two forms exist for most member functions that access instance
variables.  The first takes two arguments: the \bold{class}, \var{k} and the
value being stored.  The second form includes the \bold{entry}, \var{e} and
\bold{phase}, \var{p}. 

\begin{description}
\item[setService] \texonly{---} Set service time.\\
  \code{Server& setService( const unsigned k, const double s);}\\
  \code{Server& setService( const unsigned e, const unsigned k, const
    unsigned p, const double s);}

  Set the service time for the station to \var{s}.  

\item[setVisits] \texonly{---} Set visit ratios.\\
  \code{Server& setVisits( const unsigned k, const double v );}\\
  \code{Server& setVisits( const unsigned e, const unsigned k, const
    double v );}

  Set the visit ratio for the station to \var{v}.

\item[addVisits] \texonly{---} Add to visit ratio.\\
  \code{Server& setVisits( const unsigned e, const unsigned k, const
    double v );}

  Add \var{v} to the present value of the visit ratio for the station.

\item[clearVisits] \texonly{---} Clear visit ratios.\\
  \code{clearVistis( const unsigned k );}

  Clear visit ratios for class \var{k}.

\item[setVariance] \texonly{---} Set variance.\\
  \code{Server& setVariance( const unsigned e, const unsigned k, const
    unsigned p, const double v);}\\
  \code{Server& setServerVariance( const double v);}

  Set the variance for the entry (or station) to \var{v}.  The server
  class must be one that uses variance in its waiting time calculation.
  Otherwise, a call to these functions will generate run-time errors.
\end{description}

\subsubsection{Queries}
\begin{description}
\item[nEntries] \texonly{---} Number of Entries.\\
  \code{unsigned nEntries() const;}

  Return the number of entries for this station.

\item[S] \texonly{---} Service Time.\\
  \code{double S( const unsigned e, const unsigned k ) const;}\\
  \code{double S( const unsigned e, const unsigned k, const unsigned p ) const;}\\
  \code{double etaS( const unsigned e, const unsigned k ) const;}\\

  Return the service time.  The first form of \code{S()} returns the
  total service time over all phases.  The second form returns the
  service time for the phase \var{p}.  The final form, \code{etaS()}
  returns the service time ratioed by the number of visits (see
  \link{\code{eta()}}{sec:server-eta}).

\item[V] \texonly{---} Visit ratios.\\
  \code{double V( const unsigned e, const unsigned k ) const;}\\
  \code{double V( const unsigned k ) const;}

  Return the visit ratios by [optionally] \bold{entry} and \bold{class}.

\item[R] \texonly{---} Residence time.\\
  \code{double R( const unsigned e, const unsigned k ) const;}\\
  \code{double R( const unsigned k ) const;}\\
  \code{double R() const;}

  Return the residence time given by \math[visits times waiting]{v_{ek}
    W_{ek}}.  Residence time can be returned by \bold{entry} and
  \bold{class}.  

\item[computeMarginalQueueSize] \texonly{---} Return true if necessary\\
  \code{int computeMarginalQueueSize();}

  Return non-zero value if the marginal queue sizes need to be
  calculated for the station.  By default, marginal queue sizes are NOT
  calclated. 

  \label{sec:server-eta}
\item[eta] \texonly{---} Visit ratio fraction.\\
  \code{double eta( const unsigned e, const unsigned k ) const;}

  Return the fraction of visits by \var{e} and \var{k} to the total
  number of visits to the station.  Private member function.

\item[prOt] \texonly{---} Overtaking probability.\\
  \code{double prOt( const unsigned e, const unsigned k, const unsigned
    p) const;}

  Return the overtaking probability.  The overtaking probabilty for most
  server types is zero by default.  Overridden by subclasses that
  support phases.

\end{description}

\subsubsection{Computation}

\begin{description}
\item[wait] \texonly{---} Find waiting time. \\
  \code{double wait( const MVA& solver, const unsigned e, const unsigned k, const
    PopVector & N ) const;}

  Calculate and return the waiting time using the \link{MVA}{sec:mva}
  solver, \var{solver} for \bold{entry} \var{e} and \bold{class} \var{k}
  at a \link{population vector}{sec:vector} \var{N}.  Implemented by subclasses.

\item[interlock] \texonly{---} Find interlocked throughput.\\
  \code{double interlock( const double lambda, const unsigned k) const;}

  Return the interlocked throughput at this station for a \bold{class}
  \var{k} for a given input throughput of \var{lambda}.  This function
  is used by the MVA sovler to adjust for interlocked flows.

\end{description}

\subsubsection{Input and Output}
\label{sec:server-print}

The following functions are used to print information about the
receiver.  \var{Ostream} specifies the output stream.

\begin{description}
\item[print] \texonly{---} Print input parameters.\\
  \code{ostream& print( ostream& os ) const;}

  Print input information about this station.  \code{Print()} calls
  \code{printInput()} for all entries and classes.

\item[printExtraInfo] \texonly{---} Print bonus infor\\
  \code{ostream& printExtraInfo( ostream& output, const unsigned = 0 );}

  Print out any extra information (such as overtaking probabilities) for
  the station.  By default, there is no extra information to print.

\item[printInput] \texonly{---} Print input parameters.\\
  \code{ostream& printInput( ostream&, const unsigned, const unsigned )
    const;}

  Print out the input values for Service Time
  \link{\var{s}}{sec:server-ivars} and Visit Ratios
  \link{\var{v}}{sec:server-ivars} for \bold{entry} \var{e},
  \bold{class} \var{k}. 

\item[printResults] \texonly{---} Print results.\\
  \code{ostream& printResults( ostream& os ) const;}

  Print result information about this station.  \code{Print()} calls
  \code{printEntry()} for all entries and classes.

\item[printEntry] \texonly{---} Print results.\\
  \code{ostream& printEntry( ostream& os, const unsigned e, const
    unsigned k ) const;}

  Print Queue Length \link{\var{L}}{sec:server-ivars}, Waiting Time
  \link{\var{W}}{sec:server-ivars}, throughput and Utilization
  \link{\var{U}}{sec:server-ivars} for this station for \bold{entry}
  \var{e} and \bold{class} \var{k}.
\end{description}

\subsection{Class FCFS_Server}
\label{sec:server-fcfs}

\subsection{Class Delay_Server}
\label{sec:server-delay}

\subsection{Class HVFCFS_Server}
\label{sec:server-hvfcfs}

\subsection{Class Phased_Server}
\label{sec:server-phased}

\subsection{Class HVFCFS_Phased_Server}
\label{sec:server-hvfcfs-phased}

\subsection{Class Blended_HVFCFS_Server}
\label{sec:server-blended-hvfcfs}

\subsection{Class Blended_HVFCFS_Phased_Server}
\label{sec:server-blended-hvfcfs-phased}

\C
\C ----------------------------------------------------------------------
\C
\subsection{Class Simple_Multi_Server}
\label{sec:simple-multiserver}

This class implements \link{BCMP-type (or product form)
  multiservers}[~\Cite]{queue:reiser-80}.  As with simple First-Come,
First Served servers, the service time for each class at a given
server should be the same.  

\subsubsection{Instance Variables}
\label{sec:simple-multiserver-ivars}
\begin{verbatim}
unsigned J;                 /* Number of servers.           */
\end{verbatim}

\begin{description}
\item[M] \texonly{---} Number of stations.
\end{description}

\subsubsection{Constructors for Simple_Multi_Server}

Constructors for multiple server stations take an additional argument,
\var{copies}, that sets the number of servers for the station object.
The number of classes \var{k} and the number of entries \var{e} are
both optional.

\begin{verbatim}
    Simple_Multi_Server( const unsigned copies );
    Simple_Multi_Server( const unsigned k, const unsigned copies );
    Simple_Multi_Server( const unsigned e, const unsigned k, const unsigned copies );
\end{verbatim}

\subsubsection{Initialization}

\begin{description}
\item[initialize] \texonly{---} Initialization\\
  \code{void initialize();}

  The initialization function merely checks that the user defined at
  least one server.
\end{description}

\subsubsection{Instance Variable Access}

\begin{description}
\item[marginalProbabilitiesSize()] \texonly{---} Return the number of servers.\\
  \code{int marginalProbabilitiesSize() const;}

  Return the number of servers \math[\var{J}]{J}.

\end{description}

\subsubsection{Computation}
\begin{description}

\item[wait] \texonly{---} Waiting Time\\
  \code{double wait( const MVA&, const unsigned, const unsigned, const PopVector & ) const;}

  The following expression, from \link{Reiser and
    Lavenberg}[~\Cite]{queue:reiser-80}, equation (3.8), is used to find
  the waiting time
  \begin{iftex}
    \tex
    \begin{equation}
      W_{mk}({\bf N}) = \frac{ s_{mk} }{J_{m}} \left[ 1 + L_{mk}({\bf N} -
      e_k) + \sum_{j=0}^{J_{m} - 2} (J_{m} - 1 - j) P(j,{\bf N} - e_k) \right]
    \end{equation}
  \end{iftex}

\end{description}

\subsubsection{Input and Output}
\begin{description}
\item[print] \texonly{---} Print input data\\
  \code{ostream& print( ostream& output ) const;}

  Print information about this station.
\end{description}

\C
\C ----------------------------------------------------------------------
\C
\subsection{Class Multi_Server}
\label{sec:multiserver}

\subsubsection{Constructors for Multi_Server}

\begin{verbatim}

\end{verbatim}

\subsubsection{Computation}
\begin{description}

\item[wait] \texonly{---} Waiting Time\\
  \code{double wait( const MVA&, const unsigned, const unsigned, const PopVector & ) const;}

  The waiting time for multiservers with service times that differ for
  each chains is from \link{Eqn (19)}[~\Cite]{queue:deSouzaeSilva-87}
  (also: \link{Eqn (2.7)}[~\Cite]{queue:conway-88}).

  \begin{iftex}
    \tex
    \begin{equation}
      W_{mk}({\bf N}) = s_{mk}
      + \sum_{i=1}^{K}{\it XE}_{mki}({\bf N}) L_{mi}^{*}({\bf N} - e_k)
      + {\it XR}_{mk}({\bf N})P_{m}(J_{m},{\bf N} - e_k) 
    \end{equation}
  \end{iftex}
  
  \begin{iftex}
      \tex
    \begin{eqnarray}
      {\it XE}_{mki}({\bf N}) & = & \sum_{{\bf n} \in A_{ik}}\frac{{\it
          PS}_{mk}({\bf n},i,{\bf N})}{n_1\mu_{m1}+...+n_K\mu_{mK}}
      \label{eqn:msfcfs-xe} \\
      {\it XR}_{mk}({\bf N}) & = & \sum_{{\bf n} \in B_{k}}\frac{{\it
          PS}_{mk}({\bf n},{\bf N})}{n_1\mu_{m1}+...+n_K\mu_{mK}}
      \label{eqn:msfcfs-xr} \\
      {\it PS}_{mk}({\bf n},i,{\bf N}) & = & \frac{A_m( {\bf n}, {\bf N} -
        e_k )}{C_{mi}( {\bf N} - e_k )}
      \\
      {\it PS}_{mk}({\bf n},{\bf N}) & = & \frac{A_m( {\bf n}, {\bf N} -
        e_k )}{C_m( {\bf N} - e_k )}
      \\
      A_m({\bf n},{\bf N}-e_k) & = & \frac{J_{m}!}{n_1!n_2!...n_K!} \left[
      \prod_{x=1}^{K} F_{mx}^{n_{x}}({\bf N}-e_k) \right]
      \\
      C_{mi}({\bf N}-e_k) & = & \sum_{{\bf n} \in A_{ik}}A_m({\bf n},{\bf N}-e_k)
      \\
      C_m({\bf N}-e_k) & = & \sum_{{\bf n} \in B_{k}}A_m({\bf n},{\bf
        N}-e_k)
      \\ 
      F_{mx}({\bf N}-e_k) & = & \frac{U_{mx}({\bf
          N}-e_k)}{\sum_{y=1}^{K} U_{my}({\bf N}-e_k)}
      \label{eqn:msfcfs-f} \\
      L_{mi}^{*}({\bf N} - e_k) & = & L_{mi}({\bf N} - e_k) -
      U_{mi}({\bf N} - e_k)
    \end{eqnarray}
  \end{iftex}

  The populations for the vector {\bf n} used in the summations of the
  \var{XE} and \var{XR} terms in the waiting time expression,
  (\link{Eqns (16) and (17)}[~\Cite]{queue:deSouzaeSilva-87}), are
  generated by objects of the classes
  \link{\file{A_Iterator}}[~(\Sec\Ref)]{sec:pop-AIter} and
  \link{\file{B_iterator}}[~(\Sec\Ref)]{sec:pop-BIter} respectively.
  \begin{iftex}
    \tex
    The populations satisfy the constraints of (eqn:pop-A) and
    (eqn-pop-B). 
    \begin{eqnarray}
      A_{ik} & = & \leftbracket {\bf n} | n_1+...+n_K = J_{m}, n_j \geq 0, j =
      1,...,K, {\bf n} \leq ({\bf N} - e_k), n_i \geq 1 \rightbracket
      \label{eqn:pop-A}\\ 
      B_{k} & = & \leftbracket {\bf n} | n_1+...+n_K = J_{m}, n_j \geq 0, j =
      1,...,K, {\bf n} \leq ({\bf N} - e_k) \rightbracket
      \label{eqn:pop-B}
    \end{eqnarray}
  \end{iftex}

\item[effectiveBacklog] \texonly{---} XE term.\\
  \code{double effectiveBacklog( const MVA& solver, const PopVector& N,
    const unsigned k ) const;}

  The mean effective backlog of work in the queue when a chain \var{k}
  customer arrives, \math[\var{XE}]{XE_{mk}} (\link{Eqn
    (18)}[~\Cite]{queue:deSouzaeSilva-87}), multiplied by the number
  of customers in the queue.

  \begin{iftex}
    \tex
    \begin{displaymath}
      \sum_{i=1}^{K}{\it XE}_{mki}({\bf N}) L_{mi}^{*}({\bf N} - e_k)
    \end{displaymath}
  \end{iftex}

\item[departureTime] \texonly{---} XR term.\\
  \code{double departureTime( const MVA& solver, const PopVector& N, const unsigned k ) const;}

  The average elapsed time until the first customer leaves,
  \math[\var{XR}]{XR_{mk}} \link{(Eqn
    (16))}[~\Cite]{queue:deSouzaeSilva-87}, multiplied by the
  probability that all servers are busy,
  \math[\var{P(J,N-ek)}]{P_{m}(J_{m},{\bf N} - e_k)}.

  \begin{iftex}
    \tex
    \begin{displaymath}
      {\it XR}_{mk}({\bf N})P_{m}(J_{m},{\bf N} - e_k) 
    \end{displaymath}
  \end{iftex}

\item[sumOf_PS_k] \texonly{---} \math[Sum of A(n,N-ek)/C(n,N-ek)]{\sum
    \frac{A_m( {\bf n}, {\bf N} - e_k )}{C_m( {\bf N} - e_k )}}\\ 
  \code{double sumOf_PS_k( const MVA& solver, const PopVector& N,
    const unsigned k, PopulationIterator& next ) const;}

  Common expression to the XE and XR calculation \link{(Eqns (9) and
    (13))}[~\Cite]{queue:deSouzaeSilva-87}.  The `\var{next}' argument
  determines the values that are used in the sum.

  \begin{iftex}
    Equations~(\ref{eqn:msfcfs-xe}) and (\ref{eqn:msfcfs-xr}) are
    re-expressed as:

    \tex
    \begin{eqnarray*}
      {\it XE}_{mki}({\bf N}) & = & \frac{\displaystyle \sum_{ {\bf n} \in
        A_{ik}} \frac{ A_{m}({\bf n},{\bf N} - e_k) }{
        n_1\mu_{m1}+...+n_K\mu_{mK} } }{\displaystyle \sum_{{\bf n} \in A_{ik}}
      A_{m}({\bf n},{\bf N} - e_k) }
      \label{eqn:msfcfs-xe-new} \\
      {\it XR}_{mk}({\bf N}) & = & \frac{\displaystyle \sum_{ {\bf n} \in
        B_{k}} \frac{A_{m}({\bf n},{\bf N} - e_k) }{
        n_1\mu_{m1}+...+n_K\mu_{mK} } }{\displaystyle \sum_{{\bf n} \in B_{k}}
      A_{m}({\bf n},{\bf N} - e_k) }
      \label{eqn:msfcfs-xr-new}
    \end{eqnarray*}
  \end{iftex}

\item[A] \texonly{---} \math[\var{A}]{A_m} term \\
  \code{const MVA& solver, const PopVector& n, const PopVector& N, const unsigned k ) const;}

  The \math[\var{A}]{A_m} term \link{(Eqn
    10)}[~\Cite]{queue:deSouzaeSilva-87}.

  \begin{iftex}
    \tex
    \begin{displaymath}
      A_m({\bf n},{\bf N}-e_k) = J_{m}! \cdot \frac{\displaystyle
      \prod_{x=1}^{K} F_{mx}^{n_{x}}({\bf N}-e_k) }{\displaystyle
      \prod_{x=1}^{K} n_{x}! }
    \end{displaymath}
  \end{iftex}

\item[F] \texonly{---} Fraction of utilization \\
  \code{double F( const MVA& solver, const unsigned i, const PopVector& N, const unsigned k ) const;}

  The fraction of class \var{i} utilization at the server \link{(Eqn
  (11))}[~\Cite]{queue:deSouzaeSilva-87}).

  \begin{iftex}
    \tex
    \begin{displaymath}
      F_{mx}({\bf N}-e_k) = \frac{ U_{mx}({\bf N}-e_k) }{
        \sum_{y=1}^{K} U_{my}({\bf N}-e_k) }
    \end{displaymath}
  \end{iftex}

\item[meanMinimumService] \texonly{---} Mean minimum service time\\
  \code{double meanMinimumService( const PopVector& N ) const;}

  The elapsed time for the arrival of this customer until the next
  departure \link{Eqn (15)}[~\Cite]{queue:deSouzaeSilva-87}.  It is
  simply the minimum of the service times of customers in service for
  exponentially distributed service times:
  
  \begin{iftex}
    \tex
    \begin{displaymath}
      \frac{ 1 }{ \sum_{y=1}^{K} \frac{ n_{k} }{ s_{k} } }
    \end{displaymath}
  \end{iftex}

\end{description}


\subsubsection{Input and Output}

\begin{description}
\item[print] \texonly{---} Print input data\\
  \code{ostream& print( ostream& output ) const;}

  Print information about this station.
\end{description}


\subsection{Class MOL Multiserver}
\label{sec:mol-multiserver}

\subsubsection{Instance Variables}
\label{sec:mol-multiserver-ivars}
\begin{verbatim}

\end{verbatim}

\begin{description}
\item[wait] \texonly{---} Waiting Time\\
  \code{double wait( const MVA&, const unsigned, const unsigned, const
    PopVector & ) const;}

  The waiting time expression for...

\end{description}

\subsubsection{Constructors for MOL Multiserver}

\begin{verbatim}

\end{verbatim}

\subsubsection{Initialization}
\subsubsection{Copying}
\subsubsection{Instance Variable Access}
\subsubsection{Queries}
\subsubsection{Computation}
\subsubsection{Input and Output}


\C Local Variables: 
\C mode: latex
\C TeX-master: "lqns"
\C End: 

    \caption{Server class hierarchy}
    \H \htmlimage{server.gif}
  \end{center}
\end{figure}

\htmlrule
\C
\C ----------------------------------------------------------------------
\C
\subsection{Class Server  --- Abstract Superclass}
\subsubsection{Instance variables}
\label{sec:server-ivars}

\begin{verbatim}
public:
double *W;                  /* Waiting time per visit.      */
unsigned index;             /* Not used locally.            */
Interlock * IL;             /* Interlock stuff.             */

protected:             
const unsigned E;           /* Number of entries.           */
const unsigned K;           /* Number of Classes.           */
double ***s;                /* Service Time per phase.      */
double ***v;                /* Visit ratios per phase.      */
\end{verbatim}

\begin{description}
\item[E] \texonly{---} Number of \underline{E}ntries.
\item[IL] \texonly{---} Interlocked throughput.  Dimensioned by \bold{class}.
\item[K] \texonly{---} Number of Classes.
\item[s] \texonly{---} \underline{S}ervice time per visit.  Dimensioned by
  \bold{entry}, \bold{class} and \bold{phase}.
\item[v] \texonly{---} \underline{V}isit ratios. Dimensioned by
  \bold{entry}, \bold{class} and \bold{phase}.
\item[W] \texonly{---} \underline{W}aiting time per visit.  Dimensioned by \bold{class}.
\end{description}

\subsubsection{Constructors for Class Server}

Servers are created by calling the appropriate server type constructor
with two arguments, \var{e} and \var{k}.  The [optional] \var{e}
argument specifies the maximum number of entries for the server.  The
[optional] \var{k} argument specifies the number of classes.  If the
number of entries or classes is not specified, one is assumed.  These
values cannot be changed once the server is created.

\begin{verbatim}
Server();
Server( const unsigned k );
Server( const unsigned e, const unsigned k );
\end{verbatim}


\subsubsection{Initialization}
\label{sec:server-initialization}

\begin{description}
\item[initialize] \texonly{---} Initialization.\\
  \code{void initialize( const unsigned e, const unsigned k );}

  \code{Initialize()} is used to allocate storage for most of the arrays
  defined in this class.  The exceptions are the
  \link{populations}{sec:population} \var{L} and \var{U}; these instance
  variabels are allocated by the appropriate \link{MVA}{sec:mva} solver.

  \label{sec:server-initstep}
\item[initStep] \texonly{---} Per-MVA step initialization.\\
  \code{void initStep( const MVA& solver);}

  This method is called before the \link{MVA step}{sec:mva-step} for a
  given population.  Some server types need initialization while others
  do not.  By default, no operation is performed but is overridden by
  subclasses.

\item[check] \texonly{---} Simple model check.\\
  \code{void check();}

  Perform a simple model consistency check.
\end{description}

\subsubsection{Instance Variable Access}

Two forms exist for most member functions that access instance
variables.  The first takes two arguments: the \bold{class}, \var{k} and the
value being stored.  The second form includes the \bold{entry}, \var{e} and
\bold{phase}, \var{p}. 

\begin{description}
\item[setService] \texonly{---} Set service time.\\
  \code{Server& setService( const unsigned k, const double s);}\\
  \code{Server& setService( const unsigned e, const unsigned k, const
    unsigned p, const double s);}

  Set the service time for the station to \var{s}.  

\item[setVisits] \texonly{---} Set visit ratios.\\
  \code{Server& setVisits( const unsigned k, const double v );}\\
  \code{Server& setVisits( const unsigned e, const unsigned k, const
    double v );}

  Set the visit ratio for the station to \var{v}.

\item[addVisits] \texonly{---} Add to visit ratio.\\
  \code{Server& setVisits( const unsigned e, const unsigned k, const
    double v );}

  Add \var{v} to the present value of the visit ratio for the station.

\item[clearVisits] \texonly{---} Clear visit ratios.\\
  \code{clearVistis( const unsigned k );}

  Clear visit ratios for class \var{k}.

\item[setVariance] \texonly{---} Set variance.\\
  \code{Server& setVariance( const unsigned e, const unsigned k, const
    unsigned p, const double v);}\\
  \code{Server& setServerVariance( const double v);}

  Set the variance for the entry (or station) to \var{v}.  The server
  class must be one that uses variance in its waiting time calculation.
  Otherwise, a call to these functions will generate run-time errors.
\end{description}

\subsubsection{Queries}
\begin{description}
\item[nEntries] \texonly{---} Number of Entries.\\
  \code{unsigned nEntries() const;}

  Return the number of entries for this station.

\item[S] \texonly{---} Service Time.\\
  \code{double S( const unsigned e, const unsigned k ) const;}\\
  \code{double S( const unsigned e, const unsigned k, const unsigned p ) const;}\\
  \code{double etaS( const unsigned e, const unsigned k ) const;}\\

  Return the service time.  The first form of \code{S()} returns the
  total service time over all phases.  The second form returns the
  service time for the phase \var{p}.  The final form, \code{etaS()}
  returns the service time ratioed by the number of visits (see
  \link{\code{eta()}}{sec:server-eta}).

\item[V] \texonly{---} Visit ratios.\\
  \code{double V( const unsigned e, const unsigned k ) const;}\\
  \code{double V( const unsigned k ) const;}

  Return the visit ratios by [optionally] \bold{entry} and \bold{class}.

\item[R] \texonly{---} Residence time.\\
  \code{double R( const unsigned e, const unsigned k ) const;}\\
  \code{double R( const unsigned k ) const;}\\
  \code{double R() const;}

  Return the residence time given by \math[visits times waiting]{v_{ek}
    W_{ek}}.  Residence time can be returned by \bold{entry} and
  \bold{class}.  

\item[computeMarginalQueueSize] \texonly{---} Return true if necessary\\
  \code{int computeMarginalQueueSize();}

  Return non-zero value if the marginal queue sizes need to be
  calculated for the station.  By default, marginal queue sizes are NOT
  calclated. 

  \label{sec:server-eta}
\item[eta] \texonly{---} Visit ratio fraction.\\
  \code{double eta( const unsigned e, const unsigned k ) const;}

  Return the fraction of visits by \var{e} and \var{k} to the total
  number of visits to the station.  Private member function.

\item[prOt] \texonly{---} Overtaking probability.\\
  \code{double prOt( const unsigned e, const unsigned k, const unsigned
    p) const;}

  Return the overtaking probability.  The overtaking probabilty for most
  server types is zero by default.  Overridden by subclasses that
  support phases.

\end{description}

\subsubsection{Computation}

\begin{description}
\item[wait] \texonly{---} Find waiting time. \\
  \code{double wait( const MVA& solver, const unsigned e, const unsigned k, const
    PopVector & N ) const;}

  Calculate and return the waiting time using the \link{MVA}{sec:mva}
  solver, \var{solver} for \bold{entry} \var{e} and \bold{class} \var{k}
  at a \link{population vector}{sec:vector} \var{N}.  Implemented by subclasses.

\item[interlock] \texonly{---} Find interlocked throughput.\\
  \code{double interlock( const double lambda, const unsigned k) const;}

  Return the interlocked throughput at this station for a \bold{class}
  \var{k} for a given input throughput of \var{lambda}.  This function
  is used by the MVA sovler to adjust for interlocked flows.

\end{description}

\subsubsection{Input and Output}
\label{sec:server-print}

The following functions are used to print information about the
receiver.  \var{Ostream} specifies the output stream.

\begin{description}
\item[print] \texonly{---} Print input parameters.\\
  \code{ostream& print( ostream& os ) const;}

  Print input information about this station.  \code{Print()} calls
  \code{printInput()} for all entries and classes.

\item[printExtraInfo] \texonly{---} Print bonus infor\\
  \code{ostream& printExtraInfo( ostream& output, const unsigned = 0 );}

  Print out any extra information (such as overtaking probabilities) for
  the station.  By default, there is no extra information to print.

\item[printInput] \texonly{---} Print input parameters.\\
  \code{ostream& printInput( ostream&, const unsigned, const unsigned )
    const;}

  Print out the input values for Service Time
  \link{\var{s}}{sec:server-ivars} and Visit Ratios
  \link{\var{v}}{sec:server-ivars} for \bold{entry} \var{e},
  \bold{class} \var{k}. 

\item[printResults] \texonly{---} Print results.\\
  \code{ostream& printResults( ostream& os ) const;}

  Print result information about this station.  \code{Print()} calls
  \code{printEntry()} for all entries and classes.

\item[printEntry] \texonly{---} Print results.\\
  \code{ostream& printEntry( ostream& os, const unsigned e, const
    unsigned k ) const;}

  Print Queue Length \link{\var{L}}{sec:server-ivars}, Waiting Time
  \link{\var{W}}{sec:server-ivars}, throughput and Utilization
  \link{\var{U}}{sec:server-ivars} for this station for \bold{entry}
  \var{e} and \bold{class} \var{k}.
\end{description}

\subsection{Class FCFS_Server}
\label{sec:server-fcfs}

\subsection{Class Delay_Server}
\label{sec:server-delay}

\subsection{Class HVFCFS_Server}
\label{sec:server-hvfcfs}

\subsection{Class Phased_Server}
\label{sec:server-phased}

\subsection{Class HVFCFS_Phased_Server}
\label{sec:server-hvfcfs-phased}

\subsection{Class Blended_HVFCFS_Server}
\label{sec:server-blended-hvfcfs}

\subsection{Class Blended_HVFCFS_Phased_Server}
\label{sec:server-blended-hvfcfs-phased}

\C
\C ----------------------------------------------------------------------
\C
\subsection{Class Simple_Multi_Server}
\label{sec:simple-multiserver}

This class implements \link{BCMP-type (or product form)
  multiservers}[~\Cite]{queue:reiser-80}.  As with simple First-Come,
First Served servers, the service time for each class at a given
server should be the same.  

\subsubsection{Instance Variables}
\label{sec:simple-multiserver-ivars}
\begin{verbatim}
unsigned J;                 /* Number of servers.           */
\end{verbatim}

\begin{description}
\item[M] \texonly{---} Number of stations.
\end{description}

\subsubsection{Constructors for Simple_Multi_Server}

Constructors for multiple server stations take an additional argument,
\var{copies}, that sets the number of servers for the station object.
The number of classes \var{k} and the number of entries \var{e} are
both optional.

\begin{verbatim}
    Simple_Multi_Server( const unsigned copies );
    Simple_Multi_Server( const unsigned k, const unsigned copies );
    Simple_Multi_Server( const unsigned e, const unsigned k, const unsigned copies );
\end{verbatim}

\subsubsection{Initialization}

\begin{description}
\item[initialize] \texonly{---} Initialization\\
  \code{void initialize();}

  The initialization function merely checks that the user defined at
  least one server.
\end{description}

\subsubsection{Instance Variable Access}

\begin{description}
\item[marginalProbabilitiesSize()] \texonly{---} Return the number of servers.\\
  \code{int marginalProbabilitiesSize() const;}

  Return the number of servers \math[\var{J}]{J}.

\end{description}

\subsubsection{Computation}
\begin{description}

\item[wait] \texonly{---} Waiting Time\\
  \code{double wait( const MVA&, const unsigned, const unsigned, const PopVector & ) const;}

  The following expression, from \link{Reiser and
    Lavenberg}[~\Cite]{queue:reiser-80}, equation (3.8), is used to find
  the waiting time
  \begin{iftex}
    \tex
    \begin{equation}
      W_{mk}({\bf N}) = \frac{ s_{mk} }{J_{m}} \left[ 1 + L_{mk}({\bf N} -
      e_k) + \sum_{j=0}^{J_{m} - 2} (J_{m} - 1 - j) P(j,{\bf N} - e_k) \right]
    \end{equation}
  \end{iftex}

\end{description}

\subsubsection{Input and Output}
\begin{description}
\item[print] \texonly{---} Print input data\\
  \code{ostream& print( ostream& output ) const;}

  Print information about this station.
\end{description}

\C
\C ----------------------------------------------------------------------
\C
\subsection{Class Multi_Server}
\label{sec:multiserver}

\subsubsection{Constructors for Multi_Server}

\begin{verbatim}

\end{verbatim}

\subsubsection{Computation}
\begin{description}

\item[wait] \texonly{---} Waiting Time\\
  \code{double wait( const MVA&, const unsigned, const unsigned, const PopVector & ) const;}

  The waiting time for multiservers with service times that differ for
  each chains is from \link{Eqn (19)}[~\Cite]{queue:deSouzaeSilva-87}
  (also: \link{Eqn (2.7)}[~\Cite]{queue:conway-88}).

  \begin{iftex}
    \tex
    \begin{equation}
      W_{mk}({\bf N}) = s_{mk}
      + \sum_{i=1}^{K}{\it XE}_{mki}({\bf N}) L_{mi}^{*}({\bf N} - e_k)
      + {\it XR}_{mk}({\bf N})P_{m}(J_{m},{\bf N} - e_k) 
    \end{equation}
  \end{iftex}
  
  \begin{iftex}
      \tex
    \begin{eqnarray}
      {\it XE}_{mki}({\bf N}) & = & \sum_{{\bf n} \in A_{ik}}\frac{{\it
          PS}_{mk}({\bf n},i,{\bf N})}{n_1\mu_{m1}+...+n_K\mu_{mK}}
      \label{eqn:msfcfs-xe} \\
      {\it XR}_{mk}({\bf N}) & = & \sum_{{\bf n} \in B_{k}}\frac{{\it
          PS}_{mk}({\bf n},{\bf N})}{n_1\mu_{m1}+...+n_K\mu_{mK}}
      \label{eqn:msfcfs-xr} \\
      {\it PS}_{mk}({\bf n},i,{\bf N}) & = & \frac{A_m( {\bf n}, {\bf N} -
        e_k )}{C_{mi}( {\bf N} - e_k )}
      \\
      {\it PS}_{mk}({\bf n},{\bf N}) & = & \frac{A_m( {\bf n}, {\bf N} -
        e_k )}{C_m( {\bf N} - e_k )}
      \\
      A_m({\bf n},{\bf N}-e_k) & = & \frac{J_{m}!}{n_1!n_2!...n_K!} \left[
      \prod_{x=1}^{K} F_{mx}^{n_{x}}({\bf N}-e_k) \right]
      \\
      C_{mi}({\bf N}-e_k) & = & \sum_{{\bf n} \in A_{ik}}A_m({\bf n},{\bf N}-e_k)
      \\
      C_m({\bf N}-e_k) & = & \sum_{{\bf n} \in B_{k}}A_m({\bf n},{\bf
        N}-e_k)
      \\ 
      F_{mx}({\bf N}-e_k) & = & \frac{U_{mx}({\bf
          N}-e_k)}{\sum_{y=1}^{K} U_{my}({\bf N}-e_k)}
      \label{eqn:msfcfs-f} \\
      L_{mi}^{*}({\bf N} - e_k) & = & L_{mi}({\bf N} - e_k) -
      U_{mi}({\bf N} - e_k)
    \end{eqnarray}
  \end{iftex}

  The populations for the vector {\bf n} used in the summations of the
  \var{XE} and \var{XR} terms in the waiting time expression,
  (\link{Eqns (16) and (17)}[~\Cite]{queue:deSouzaeSilva-87}), are
  generated by objects of the classes
  \link{\file{A_Iterator}}[~(\Sec\Ref)]{sec:pop-AIter} and
  \link{\file{B_iterator}}[~(\Sec\Ref)]{sec:pop-BIter} respectively.
  \begin{iftex}
    \tex
    The populations satisfy the constraints of (eqn:pop-A) and
    (eqn-pop-B). 
    \begin{eqnarray}
      A_{ik} & = & \leftbracket {\bf n} | n_1+...+n_K = J_{m}, n_j \geq 0, j =
      1,...,K, {\bf n} \leq ({\bf N} - e_k), n_i \geq 1 \rightbracket
      \label{eqn:pop-A}\\ 
      B_{k} & = & \leftbracket {\bf n} | n_1+...+n_K = J_{m}, n_j \geq 0, j =
      1,...,K, {\bf n} \leq ({\bf N} - e_k) \rightbracket
      \label{eqn:pop-B}
    \end{eqnarray}
  \end{iftex}

\item[effectiveBacklog] \texonly{---} XE term.\\
  \code{double effectiveBacklog( const MVA& solver, const PopVector& N,
    const unsigned k ) const;}

  The mean effective backlog of work in the queue when a chain \var{k}
  customer arrives, \math[\var{XE}]{XE_{mk}} (\link{Eqn
    (18)}[~\Cite]{queue:deSouzaeSilva-87}), multiplied by the number
  of customers in the queue.

  \begin{iftex}
    \tex
    \begin{displaymath}
      \sum_{i=1}^{K}{\it XE}_{mki}({\bf N}) L_{mi}^{*}({\bf N} - e_k)
    \end{displaymath}
  \end{iftex}

\item[departureTime] \texonly{---} XR term.\\
  \code{double departureTime( const MVA& solver, const PopVector& N, const unsigned k ) const;}

  The average elapsed time until the first customer leaves,
  \math[\var{XR}]{XR_{mk}} \link{(Eqn
    (16))}[~\Cite]{queue:deSouzaeSilva-87}, multiplied by the
  probability that all servers are busy,
  \math[\var{P(J,N-ek)}]{P_{m}(J_{m},{\bf N} - e_k)}.

  \begin{iftex}
    \tex
    \begin{displaymath}
      {\it XR}_{mk}({\bf N})P_{m}(J_{m},{\bf N} - e_k) 
    \end{displaymath}
  \end{iftex}

\item[sumOf_PS_k] \texonly{---} \math[Sum of A(n,N-ek)/C(n,N-ek)]{\sum
    \frac{A_m( {\bf n}, {\bf N} - e_k )}{C_m( {\bf N} - e_k )}}\\ 
  \code{double sumOf_PS_k( const MVA& solver, const PopVector& N,
    const unsigned k, PopulationIterator& next ) const;}

  Common expression to the XE and XR calculation \link{(Eqns (9) and
    (13))}[~\Cite]{queue:deSouzaeSilva-87}.  The `\var{next}' argument
  determines the values that are used in the sum.

  \begin{iftex}
    Equations~(\ref{eqn:msfcfs-xe}) and (\ref{eqn:msfcfs-xr}) are
    re-expressed as:

    \tex
    \begin{eqnarray*}
      {\it XE}_{mki}({\bf N}) & = & \frac{\displaystyle \sum_{ {\bf n} \in
        A_{ik}} \frac{ A_{m}({\bf n},{\bf N} - e_k) }{
        n_1\mu_{m1}+...+n_K\mu_{mK} } }{\displaystyle \sum_{{\bf n} \in A_{ik}}
      A_{m}({\bf n},{\bf N} - e_k) }
      \label{eqn:msfcfs-xe-new} \\
      {\it XR}_{mk}({\bf N}) & = & \frac{\displaystyle \sum_{ {\bf n} \in
        B_{k}} \frac{A_{m}({\bf n},{\bf N} - e_k) }{
        n_1\mu_{m1}+...+n_K\mu_{mK} } }{\displaystyle \sum_{{\bf n} \in B_{k}}
      A_{m}({\bf n},{\bf N} - e_k) }
      \label{eqn:msfcfs-xr-new}
    \end{eqnarray*}
  \end{iftex}

\item[A] \texonly{---} \math[\var{A}]{A_m} term \\
  \code{const MVA& solver, const PopVector& n, const PopVector& N, const unsigned k ) const;}

  The \math[\var{A}]{A_m} term \link{(Eqn
    10)}[~\Cite]{queue:deSouzaeSilva-87}.

  \begin{iftex}
    \tex
    \begin{displaymath}
      A_m({\bf n},{\bf N}-e_k) = J_{m}! \cdot \frac{\displaystyle
      \prod_{x=1}^{K} F_{mx}^{n_{x}}({\bf N}-e_k) }{\displaystyle
      \prod_{x=1}^{K} n_{x}! }
    \end{displaymath}
  \end{iftex}

\item[F] \texonly{---} Fraction of utilization \\
  \code{double F( const MVA& solver, const unsigned i, const PopVector& N, const unsigned k ) const;}

  The fraction of class \var{i} utilization at the server \link{(Eqn
  (11))}[~\Cite]{queue:deSouzaeSilva-87}).

  \begin{iftex}
    \tex
    \begin{displaymath}
      F_{mx}({\bf N}-e_k) = \frac{ U_{mx}({\bf N}-e_k) }{
        \sum_{y=1}^{K} U_{my}({\bf N}-e_k) }
    \end{displaymath}
  \end{iftex}

\item[meanMinimumService] \texonly{---} Mean minimum service time\\
  \code{double meanMinimumService( const PopVector& N ) const;}

  The elapsed time for the arrival of this customer until the next
  departure \link{Eqn (15)}[~\Cite]{queue:deSouzaeSilva-87}.  It is
  simply the minimum of the service times of customers in service for
  exponentially distributed service times:
  
  \begin{iftex}
    \tex
    \begin{displaymath}
      \frac{ 1 }{ \sum_{y=1}^{K} \frac{ n_{k} }{ s_{k} } }
    \end{displaymath}
  \end{iftex}

\end{description}


\subsubsection{Input and Output}

\begin{description}
\item[print] \texonly{---} Print input data\\
  \code{ostream& print( ostream& output ) const;}

  Print information about this station.
\end{description}


\subsection{Class MOL Multiserver}
\label{sec:mol-multiserver}

\subsubsection{Instance Variables}
\label{sec:mol-multiserver-ivars}
\begin{verbatim}

\end{verbatim}

\begin{description}
\item[wait] \texonly{---} Waiting Time\\
  \code{double wait( const MVA&, const unsigned, const unsigned, const
    PopVector & ) const;}

  The waiting time expression for...

\end{description}

\subsubsection{Constructors for MOL Multiserver}

\begin{verbatim}

\end{verbatim}

\subsubsection{Initialization}
\subsubsection{Copying}
\subsubsection{Instance Variable Access}
\subsubsection{Queries}
\subsubsection{Computation}
\subsubsection{Input and Output}


\C Local Variables: 
\C mode: latex
\C TeX-master: "lqns"
\C End: 

    \caption{Server class hierarchy}
    \H \htmlimage{server.gif}
  \end{center}
\end{figure}

\htmlrule
\C
\C ----------------------------------------------------------------------
\C
\subsection{Class Server  --- Abstract Superclass}
\subsubsection{Instance variables}
\label{sec:server-ivars}

\begin{verbatim}
public:
double *W;                  /* Waiting time per visit.      */
unsigned index;             /* Not used locally.            */
Interlock * IL;             /* Interlock stuff.             */

protected:             
const unsigned E;           /* Number of entries.           */
const unsigned K;           /* Number of Classes.           */
double ***s;                /* Service Time per phase.      */
double ***v;                /* Visit ratios per phase.      */
\end{verbatim}

\begin{description}
\item[E] \texonly{---} Number of \underline{E}ntries.
\item[IL] \texonly{---} Interlocked throughput.  Dimensioned by \bold{class}.
\item[K] \texonly{---} Number of Classes.
\item[s] \texonly{---} \underline{S}ervice time per visit.  Dimensioned by
  \bold{entry}, \bold{class} and \bold{phase}.
\item[v] \texonly{---} \underline{V}isit ratios. Dimensioned by
  \bold{entry}, \bold{class} and \bold{phase}.
\item[W] \texonly{---} \underline{W}aiting time per visit.  Dimensioned by \bold{class}.
\end{description}

\subsubsection{Constructors for Class Server}

Servers are created by calling the appropriate server type constructor
with two arguments, \var{e} and \var{k}.  The [optional] \var{e}
argument specifies the maximum number of entries for the server.  The
[optional] \var{k} argument specifies the number of classes.  If the
number of entries or classes is not specified, one is assumed.  These
values cannot be changed once the server is created.

\begin{verbatim}
Server();
Server( const unsigned k );
Server( const unsigned e, const unsigned k );
\end{verbatim}


\subsubsection{Initialization}
\label{sec:server-initialization}

\begin{description}
\item[initialize] \texonly{---} Initialization.\\
  \code{void initialize( const unsigned e, const unsigned k );}

  \code{Initialize()} is used to allocate storage for most of the arrays
  defined in this class.  The exceptions are the
  \link{populations}{sec:population} \var{L} and \var{U}; these instance
  variabels are allocated by the appropriate \link{MVA}{sec:mva} solver.

  \label{sec:server-initstep}
\item[initStep] \texonly{---} Per-MVA step initialization.\\
  \code{void initStep( const MVA& solver);}

  This method is called before the \link{MVA step}{sec:mva-step} for a
  given population.  Some server types need initialization while others
  do not.  By default, no operation is performed but is overridden by
  subclasses.

\item[check] \texonly{---} Simple model check.\\
  \code{void check();}

  Perform a simple model consistency check.
\end{description}

\subsubsection{Instance Variable Access}

Two forms exist for most member functions that access instance
variables.  The first takes two arguments: the \bold{class}, \var{k} and the
value being stored.  The second form includes the \bold{entry}, \var{e} and
\bold{phase}, \var{p}. 

\begin{description}
\item[setService] \texonly{---} Set service time.\\
  \code{Server& setService( const unsigned k, const double s);}\\
  \code{Server& setService( const unsigned e, const unsigned k, const
    unsigned p, const double s);}

  Set the service time for the station to \var{s}.  

\item[setVisits] \texonly{---} Set visit ratios.\\
  \code{Server& setVisits( const unsigned k, const double v );}\\
  \code{Server& setVisits( const unsigned e, const unsigned k, const
    double v );}

  Set the visit ratio for the station to \var{v}.

\item[addVisits] \texonly{---} Add to visit ratio.\\
  \code{Server& setVisits( const unsigned e, const unsigned k, const
    double v );}

  Add \var{v} to the present value of the visit ratio for the station.

\item[clearVisits] \texonly{---} Clear visit ratios.\\
  \code{clearVistis( const unsigned k );}

  Clear visit ratios for class \var{k}.

\item[setVariance] \texonly{---} Set variance.\\
  \code{Server& setVariance( const unsigned e, const unsigned k, const
    unsigned p, const double v);}\\
  \code{Server& setServerVariance( const double v);}

  Set the variance for the entry (or station) to \var{v}.  The server
  class must be one that uses variance in its waiting time calculation.
  Otherwise, a call to these functions will generate run-time errors.
\end{description}

\subsubsection{Queries}
\begin{description}
\item[nEntries] \texonly{---} Number of Entries.\\
  \code{unsigned nEntries() const;}

  Return the number of entries for this station.

\item[S] \texonly{---} Service Time.\\
  \code{double S( const unsigned e, const unsigned k ) const;}\\
  \code{double S( const unsigned e, const unsigned k, const unsigned p ) const;}\\
  \code{double etaS( const unsigned e, const unsigned k ) const;}\\

  Return the service time.  The first form of \code{S()} returns the
  total service time over all phases.  The second form returns the
  service time for the phase \var{p}.  The final form, \code{etaS()}
  returns the service time ratioed by the number of visits (see
  \link{\code{eta()}}{sec:server-eta}).

\item[V] \texonly{---} Visit ratios.\\
  \code{double V( const unsigned e, const unsigned k ) const;}\\
  \code{double V( const unsigned k ) const;}

  Return the visit ratios by [optionally] \bold{entry} and \bold{class}.

\item[R] \texonly{---} Residence time.\\
  \code{double R( const unsigned e, const unsigned k ) const;}\\
  \code{double R( const unsigned k ) const;}\\
  \code{double R() const;}

  Return the residence time given by \math[visits times waiting]{v_{ek}
    W_{ek}}.  Residence time can be returned by \bold{entry} and
  \bold{class}.  

\item[computeMarginalQueueSize] \texonly{---} Return true if necessary\\
  \code{int computeMarginalQueueSize();}

  Return non-zero value if the marginal queue sizes need to be
  calculated for the station.  By default, marginal queue sizes are NOT
  calclated. 

  \label{sec:server-eta}
\item[eta] \texonly{---} Visit ratio fraction.\\
  \code{double eta( const unsigned e, const unsigned k ) const;}

  Return the fraction of visits by \var{e} and \var{k} to the total
  number of visits to the station.  Private member function.

\item[prOt] \texonly{---} Overtaking probability.\\
  \code{double prOt( const unsigned e, const unsigned k, const unsigned
    p) const;}

  Return the overtaking probability.  The overtaking probabilty for most
  server types is zero by default.  Overridden by subclasses that
  support phases.

\end{description}

\subsubsection{Computation}

\begin{description}
\item[wait] \texonly{---} Find waiting time. \\
  \code{double wait( const MVA& solver, const unsigned e, const unsigned k, const
    PopVector & N ) const;}

  Calculate and return the waiting time using the \link{MVA}{sec:mva}
  solver, \var{solver} for \bold{entry} \var{e} and \bold{class} \var{k}
  at a \link{population vector}{sec:vector} \var{N}.  Implemented by subclasses.

\item[interlock] \texonly{---} Find interlocked throughput.\\
  \code{double interlock( const double lambda, const unsigned k) const;}

  Return the interlocked throughput at this station for a \bold{class}
  \var{k} for a given input throughput of \var{lambda}.  This function
  is used by the MVA sovler to adjust for interlocked flows.

\end{description}

\subsubsection{Input and Output}
\label{sec:server-print}

The following functions are used to print information about the
receiver.  \var{Ostream} specifies the output stream.

\begin{description}
\item[print] \texonly{---} Print input parameters.\\
  \code{ostream& print( ostream& os ) const;}

  Print input information about this station.  \code{Print()} calls
  \code{printInput()} for all entries and classes.

\item[printExtraInfo] \texonly{---} Print bonus infor\\
  \code{ostream& printExtraInfo( ostream& output, const unsigned = 0 );}

  Print out any extra information (such as overtaking probabilities) for
  the station.  By default, there is no extra information to print.

\item[printInput] \texonly{---} Print input parameters.\\
  \code{ostream& printInput( ostream&, const unsigned, const unsigned )
    const;}

  Print out the input values for Service Time
  \link{\var{s}}{sec:server-ivars} and Visit Ratios
  \link{\var{v}}{sec:server-ivars} for \bold{entry} \var{e},
  \bold{class} \var{k}. 

\item[printResults] \texonly{---} Print results.\\
  \code{ostream& printResults( ostream& os ) const;}

  Print result information about this station.  \code{Print()} calls
  \code{printEntry()} for all entries and classes.

\item[printEntry] \texonly{---} Print results.\\
  \code{ostream& printEntry( ostream& os, const unsigned e, const
    unsigned k ) const;}

  Print Queue Length \link{\var{L}}{sec:server-ivars}, Waiting Time
  \link{\var{W}}{sec:server-ivars}, throughput and Utilization
  \link{\var{U}}{sec:server-ivars} for this station for \bold{entry}
  \var{e} and \bold{class} \var{k}.
\end{description}

\subsection{Class FCFS_Server}
\label{sec:server-fcfs}

\subsection{Class Delay_Server}
\label{sec:server-delay}

\subsection{Class HVFCFS_Server}
\label{sec:server-hvfcfs}

\subsection{Class Phased_Server}
\label{sec:server-phased}

\subsection{Class HVFCFS_Phased_Server}
\label{sec:server-hvfcfs-phased}

\subsection{Class Blended_HVFCFS_Server}
\label{sec:server-blended-hvfcfs}

\subsection{Class Blended_HVFCFS_Phased_Server}
\label{sec:server-blended-hvfcfs-phased}

\C
\C ----------------------------------------------------------------------
\C
\subsection{Class Simple_Multi_Server}
\label{sec:simple-multiserver}

This class implements \link{BCMP-type (or product form)
  multiservers}[~\Cite]{queue:reiser-80}.  As with simple First-Come,
First Served servers, the service time for each class at a given
server should be the same.  

\subsubsection{Instance Variables}
\label{sec:simple-multiserver-ivars}
\begin{verbatim}
unsigned J;                 /* Number of servers.           */
\end{verbatim}

\begin{description}
\item[M] \texonly{---} Number of stations.
\end{description}

\subsubsection{Constructors for Simple_Multi_Server}

Constructors for multiple server stations take an additional argument,
\var{copies}, that sets the number of servers for the station object.
The number of classes \var{k} and the number of entries \var{e} are
both optional.

\begin{verbatim}
    Simple_Multi_Server( const unsigned copies );
    Simple_Multi_Server( const unsigned k, const unsigned copies );
    Simple_Multi_Server( const unsigned e, const unsigned k, const unsigned copies );
\end{verbatim}

\subsubsection{Initialization}

\begin{description}
\item[initialize] \texonly{---} Initialization\\
  \code{void initialize();}

  The initialization function merely checks that the user defined at
  least one server.
\end{description}

\subsubsection{Instance Variable Access}

\begin{description}
\item[marginalProbabilitiesSize()] \texonly{---} Return the number of servers.\\
  \code{int marginalProbabilitiesSize() const;}

  Return the number of servers \math[\var{J}]{J}.

\end{description}

\subsubsection{Computation}
\begin{description}

\item[wait] \texonly{---} Waiting Time\\
  \code{double wait( const MVA&, const unsigned, const unsigned, const PopVector & ) const;}

  The following expression, from \link{Reiser and
    Lavenberg}[~\Cite]{queue:reiser-80}, equation (3.8), is used to find
  the waiting time
  \begin{iftex}
    \tex
    \begin{equation}
      W_{mk}({\bf N}) = \frac{ s_{mk} }{J_{m}} \left[ 1 + L_{mk}({\bf N} -
      e_k) + \sum_{j=0}^{J_{m} - 2} (J_{m} - 1 - j) P(j,{\bf N} - e_k) \right]
    \end{equation}
  \end{iftex}

\end{description}

\subsubsection{Input and Output}
\begin{description}
\item[print] \texonly{---} Print input data\\
  \code{ostream& print( ostream& output ) const;}

  Print information about this station.
\end{description}

\C
\C ----------------------------------------------------------------------
\C
\subsection{Class Multi_Server}
\label{sec:multiserver}

\subsubsection{Constructors for Multi_Server}

\begin{verbatim}

\end{verbatim}

\subsubsection{Computation}
\begin{description}

\item[wait] \texonly{---} Waiting Time\\
  \code{double wait( const MVA&, const unsigned, const unsigned, const PopVector & ) const;}

  The waiting time for multiservers with service times that differ for
  each chains is from \link{Eqn (19)}[~\Cite]{queue:deSouzaeSilva-87}
  (also: \link{Eqn (2.7)}[~\Cite]{queue:conway-88}).

  \begin{iftex}
    \tex
    \begin{equation}
      W_{mk}({\bf N}) = s_{mk}
      + \sum_{i=1}^{K}{\it XE}_{mki}({\bf N}) L_{mi}^{*}({\bf N} - e_k)
      + {\it XR}_{mk}({\bf N})P_{m}(J_{m},{\bf N} - e_k) 
    \end{equation}
  \end{iftex}
  
  \begin{iftex}
      \tex
    \begin{eqnarray}
      {\it XE}_{mki}({\bf N}) & = & \sum_{{\bf n} \in A_{ik}}\frac{{\it
          PS}_{mk}({\bf n},i,{\bf N})}{n_1\mu_{m1}+...+n_K\mu_{mK}}
      \label{eqn:msfcfs-xe} \\
      {\it XR}_{mk}({\bf N}) & = & \sum_{{\bf n} \in B_{k}}\frac{{\it
          PS}_{mk}({\bf n},{\bf N})}{n_1\mu_{m1}+...+n_K\mu_{mK}}
      \label{eqn:msfcfs-xr} \\
      {\it PS}_{mk}({\bf n},i,{\bf N}) & = & \frac{A_m( {\bf n}, {\bf N} -
        e_k )}{C_{mi}( {\bf N} - e_k )}
      \\
      {\it PS}_{mk}({\bf n},{\bf N}) & = & \frac{A_m( {\bf n}, {\bf N} -
        e_k )}{C_m( {\bf N} - e_k )}
      \\
      A_m({\bf n},{\bf N}-e_k) & = & \frac{J_{m}!}{n_1!n_2!...n_K!} \left[
      \prod_{x=1}^{K} F_{mx}^{n_{x}}({\bf N}-e_k) \right]
      \\
      C_{mi}({\bf N}-e_k) & = & \sum_{{\bf n} \in A_{ik}}A_m({\bf n},{\bf N}-e_k)
      \\
      C_m({\bf N}-e_k) & = & \sum_{{\bf n} \in B_{k}}A_m({\bf n},{\bf
        N}-e_k)
      \\ 
      F_{mx}({\bf N}-e_k) & = & \frac{U_{mx}({\bf
          N}-e_k)}{\sum_{y=1}^{K} U_{my}({\bf N}-e_k)}
      \label{eqn:msfcfs-f} \\
      L_{mi}^{*}({\bf N} - e_k) & = & L_{mi}({\bf N} - e_k) -
      U_{mi}({\bf N} - e_k)
    \end{eqnarray}
  \end{iftex}

  The populations for the vector {\bf n} used in the summations of the
  \var{XE} and \var{XR} terms in the waiting time expression,
  (\link{Eqns (16) and (17)}[~\Cite]{queue:deSouzaeSilva-87}), are
  generated by objects of the classes
  \link{\file{A_Iterator}}[~(\Sec\Ref)]{sec:pop-AIter} and
  \link{\file{B_iterator}}[~(\Sec\Ref)]{sec:pop-BIter} respectively.
  \begin{iftex}
    \tex
    The populations satisfy the constraints of (eqn:pop-A) and
    (eqn-pop-B). 
    \begin{eqnarray}
      A_{ik} & = & \leftbracket {\bf n} | n_1+...+n_K = J_{m}, n_j \geq 0, j =
      1,...,K, {\bf n} \leq ({\bf N} - e_k), n_i \geq 1 \rightbracket
      \label{eqn:pop-A}\\ 
      B_{k} & = & \leftbracket {\bf n} | n_1+...+n_K = J_{m}, n_j \geq 0, j =
      1,...,K, {\bf n} \leq ({\bf N} - e_k) \rightbracket
      \label{eqn:pop-B}
    \end{eqnarray}
  \end{iftex}

\item[effectiveBacklog] \texonly{---} XE term.\\
  \code{double effectiveBacklog( const MVA& solver, const PopVector& N,
    const unsigned k ) const;}

  The mean effective backlog of work in the queue when a chain \var{k}
  customer arrives, \math[\var{XE}]{XE_{mk}} (\link{Eqn
    (18)}[~\Cite]{queue:deSouzaeSilva-87}), multiplied by the number
  of customers in the queue.

  \begin{iftex}
    \tex
    \begin{displaymath}
      \sum_{i=1}^{K}{\it XE}_{mki}({\bf N}) L_{mi}^{*}({\bf N} - e_k)
    \end{displaymath}
  \end{iftex}

\item[departureTime] \texonly{---} XR term.\\
  \code{double departureTime( const MVA& solver, const PopVector& N, const unsigned k ) const;}

  The average elapsed time until the first customer leaves,
  \math[\var{XR}]{XR_{mk}} \link{(Eqn
    (16))}[~\Cite]{queue:deSouzaeSilva-87}, multiplied by the
  probability that all servers are busy,
  \math[\var{P(J,N-ek)}]{P_{m}(J_{m},{\bf N} - e_k)}.

  \begin{iftex}
    \tex
    \begin{displaymath}
      {\it XR}_{mk}({\bf N})P_{m}(J_{m},{\bf N} - e_k) 
    \end{displaymath}
  \end{iftex}

\item[sumOf_PS_k] \texonly{---} \math[Sum of A(n,N-ek)/C(n,N-ek)]{\sum
    \frac{A_m( {\bf n}, {\bf N} - e_k )}{C_m( {\bf N} - e_k )}}\\ 
  \code{double sumOf_PS_k( const MVA& solver, const PopVector& N,
    const unsigned k, PopulationIterator& next ) const;}

  Common expression to the XE and XR calculation \link{(Eqns (9) and
    (13))}[~\Cite]{queue:deSouzaeSilva-87}.  The `\var{next}' argument
  determines the values that are used in the sum.

  \begin{iftex}
    Equations~(\ref{eqn:msfcfs-xe}) and (\ref{eqn:msfcfs-xr}) are
    re-expressed as:

    \tex
    \begin{eqnarray*}
      {\it XE}_{mki}({\bf N}) & = & \frac{\displaystyle \sum_{ {\bf n} \in
        A_{ik}} \frac{ A_{m}({\bf n},{\bf N} - e_k) }{
        n_1\mu_{m1}+...+n_K\mu_{mK} } }{\displaystyle \sum_{{\bf n} \in A_{ik}}
      A_{m}({\bf n},{\bf N} - e_k) }
      \label{eqn:msfcfs-xe-new} \\
      {\it XR}_{mk}({\bf N}) & = & \frac{\displaystyle \sum_{ {\bf n} \in
        B_{k}} \frac{A_{m}({\bf n},{\bf N} - e_k) }{
        n_1\mu_{m1}+...+n_K\mu_{mK} } }{\displaystyle \sum_{{\bf n} \in B_{k}}
      A_{m}({\bf n},{\bf N} - e_k) }
      \label{eqn:msfcfs-xr-new}
    \end{eqnarray*}
  \end{iftex}

\item[A] \texonly{---} \math[\var{A}]{A_m} term \\
  \code{const MVA& solver, const PopVector& n, const PopVector& N, const unsigned k ) const;}

  The \math[\var{A}]{A_m} term \link{(Eqn
    10)}[~\Cite]{queue:deSouzaeSilva-87}.

  \begin{iftex}
    \tex
    \begin{displaymath}
      A_m({\bf n},{\bf N}-e_k) = J_{m}! \cdot \frac{\displaystyle
      \prod_{x=1}^{K} F_{mx}^{n_{x}}({\bf N}-e_k) }{\displaystyle
      \prod_{x=1}^{K} n_{x}! }
    \end{displaymath}
  \end{iftex}

\item[F] \texonly{---} Fraction of utilization \\
  \code{double F( const MVA& solver, const unsigned i, const PopVector& N, const unsigned k ) const;}

  The fraction of class \var{i} utilization at the server \link{(Eqn
  (11))}[~\Cite]{queue:deSouzaeSilva-87}).

  \begin{iftex}
    \tex
    \begin{displaymath}
      F_{mx}({\bf N}-e_k) = \frac{ U_{mx}({\bf N}-e_k) }{
        \sum_{y=1}^{K} U_{my}({\bf N}-e_k) }
    \end{displaymath}
  \end{iftex}

\item[meanMinimumService] \texonly{---} Mean minimum service time\\
  \code{double meanMinimumService( const PopVector& N ) const;}

  The elapsed time for the arrival of this customer until the next
  departure \link{Eqn (15)}[~\Cite]{queue:deSouzaeSilva-87}.  It is
  simply the minimum of the service times of customers in service for
  exponentially distributed service times:
  
  \begin{iftex}
    \tex
    \begin{displaymath}
      \frac{ 1 }{ \sum_{y=1}^{K} \frac{ n_{k} }{ s_{k} } }
    \end{displaymath}
  \end{iftex}

\end{description}


\subsubsection{Input and Output}

\begin{description}
\item[print] \texonly{---} Print input data\\
  \code{ostream& print( ostream& output ) const;}

  Print information about this station.
\end{description}


\subsection{Class MOL Multiserver}
\label{sec:mol-multiserver}

\subsubsection{Instance Variables}
\label{sec:mol-multiserver-ivars}
\begin{verbatim}

\end{verbatim}

\begin{description}
\item[wait] \texonly{---} Waiting Time\\
  \code{double wait( const MVA&, const unsigned, const unsigned, const
    PopVector & ) const;}

  The waiting time expression for...

\end{description}

\subsubsection{Constructors for MOL Multiserver}

\begin{verbatim}

\end{verbatim}

\subsubsection{Initialization}
\subsubsection{Copying}
\subsubsection{Instance Variable Access}
\subsubsection{Queries}
\subsubsection{Computation}
\subsubsection{Input and Output}


\C Local Variables: 
\C mode: latex
\C TeX-master: "lqns"
\C End: 
