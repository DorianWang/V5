\C 
\C $Id: call.tex 5439 2005-01-02 02:30:31Z greg $
\C 
\C $Log$
\C Revision 31.0  2005/01/02 02:30:29  greg
\C Bring all files up to revision 31.
\C
\C Revision 30.0  2003/11/12 18:56:51  greg
\C Bring all files up to revision 30.0
\C
\C Revision 29.0  2003/06/13 00:48:38  greg
\C bounce version number.
\C
\C Revision 1.3  1995/01/11 21:05:12  greg
\C Revised for R12 of MVA software.
\C
\C Revision 1.2  1994/12/09  02:02:36  greg
\C update.
\C
\C Revision 1.1  1994/06/24  19:11:32  greg
\C Initial revision
\C
\C ----------------------------------------------------------------------
\section{Call}
\label{sec:call}

Call objects are used to store the arc information such as the calling
rate and the queueing delays in the SRVN model.  Arcs can be any one
of the following types:
\begin{itemize}
\item rendezvous,
\item send-no-reply, and
\item forwarding.
\end{itemize}
In general, only one call object is allocated for a connection from
one entry to another regardless of the number of defined phases and
the type of the call.

\subsubsection{Instance Variables}
\label{sec:call-ivars}
\begin{verbatim}
public:
    Probability prForward;         /* Forwarding probability.   */
    unsigned partition;            /* Used by layerizer.        */
    
protected:
    double rnvCalls[MAX_PHASES+1]; /* rendezvous.               */
    double snrCalls[MAX_PHASES+1]; /* send no reply.            */
    double wait[MAX_PHASES+1];     /* Waiting time.             */
    const Entry* source;           /* Calling entry.            */
    const Entry* realEntry;        /* to whom I am referring to */
    const Entry* fescEntry;        /* if set, reroute to me.    */
\end{verbatim}

\begin{description}
\item[source] \texonly{---} The originating \link{\file{Entry}}{sec:entry}.

\item[realEntry] \texonly{---} The destination \link{\file{Entry}}{sec:entry}.

\item[fescEntry] \texonly{---} The destination \link{\file{Entry}}{sec:entry}
  if \link{flow equivalent service centers}[ (\Sec\Ref)]{sec:hwsw} are
  employed to span layers.

\item[rnvCalls] \texonly{---} The number of rendezvous, by 
  phase of \var{source}.  Set by calls to
  \link{\code{store_rnv_data()}}{sec:entry-load}. 

\item[snrCalls] \texonly{---} The number of asynchronous requests, by
  phase of \var{source}.  Set by calls to
  \link{\code{store_snr_data()}}{sec:entry-load}. 

\item[wait] \texonly{---} The computed waiting time for one call.  Set
  by \link{\code{saveWait()}}[ (\Sec\Ref)]{sec:entry-saveWait}.

\item[prForward] \texonly{---} The forwarding probability from the
  \var{source} entry to the destination \var{realEntry}.

\item[partition] \texonly{---} The \link{\bold{partition}}[
  (\Sec\Ref)]{sec:layerize} number to which the waiting time results,
  \var{wait}, apply.
\end{description}

\subsubsection{Constructors for Call}

\begin{verbatim}
    Call();
    Call( const Entry * fromEntry, const Entry * toEntry );
    virtual ~Call();
\end{verbatim}

\subsubsection{Initialization}

\begin{description}
\item[initialize] \texonly{---} \\
\code{void initialize();}

Clear \var{rnvCalls}, \var{snrCalls} and \var{wait}.

\end{description}

\subsubsection{Copying}

\subsubsection{Instance Variable Access}

\begin{description}

\item[setSrcEntry] \texonly{---} \\
\code{void setSrcEntry( Entry * anEntry ) }

\item[setDstEntry] \texonly{---} \\
\code{void setDstEntry( Entry * anEntry ) }

\item[setFESCEntry] \texonly{---} \\
\code{void setFESCEntry( Entry * anEntry ) }

\item[ setWait] \texonly{---} \\
\code{Call& setWait( unsigned p, double value ) }

\item[owner] \texonly{---} \\
\code{const Entry * owner() }

\item[ rendezvous] \texonly{---} \\
\code{Call& rendezvous( const unsigned p, const double value );}

\item[rendezvous] \texonly{---} \\
\code{double rendezvous( const unsigned p ) }

\item[ sendNoReply] \texonly{---} \\
\code{Call& sendNoReply( const unsigned p, const double value );}

\item[sendNoReply] \texonly{---} \\
\code{double sendNoReply( const unsigned p ) }

\end{description}

\subsubsection{Queries}
\begin{description}

\item[allCalls] \texonly{---} \\
\code{double allCalls() }

\item[srcClass] \texonly{---} \\
\code{unsigned srcClass() }

\item[srcVisits] \texonly{---} \\
\code{double srcVisits() }

\item[srcTask] \texonly{---} \\
\code{const Entity * srcTask() }

\item[queueingTime] \texonly{---} \\
\code{double queueingTime( const unsigned ) }

\item[rendezvousDelay] \texonly{---} \\
\code{double rendezvousDelay( const unsigned p ) }

\item[mrlTerm] \texonly{---} \\
\code{double mrlTerm( const unsigned p, const double x ) }

\item[isForwardedCall] \texonly{---} \\
\code{boolean isForwardedCall() }

\end{description}

\subsubsection{Computation}

\subsubsection{Input and Output}
\begin{description}

\item[ print] \texonly{---} \\
\code{ostream& print( ostream& = cout ) }

\item[ printDelays] \texonly{---} \\
\code{ostream& printDelays( ostream& output, const unsigned np, const
  double rnv[], const double fwd[] ) }

\end{description}

\subsubsection{Proxies}

The following set of functions re-route the request to the
\link{\var{realEntry}}{sec:call-ivars} or the
\link{\var{fescEntry}}{sec:call-ivars} as necessary.  Refer to class
\link{\file{Entry}}[ (\Sec\Ref)]{sec:entry} for their function.  Also
note, these functions are defined in \file{entry.h} so that they can
be in-lined.

\begin{description}
\item[entry] \texonly{---} \\
\code{const Entry * entry() }

\item[name] \texonly{---} \\
\code{const char * name() }

\item[task] \texonly{---} \\
\code{const Entity * task() }

\item[station] \texonly{---} \\
\code{Server * station() }

\item[index] \texonly{---} \\
\code{short index() }

\item[variance] \texonly{---} \\
\code{double variance( const unsigned phase ) }

\item[type] \texonly{---} \\
\code{phase_type phaseTypeFlag( const unsigned phase ) }

\item[serviceTime] \texonly{---} \\
\code{double serviceTime( const unsigned phase ) }

\item[elapsedTime] \texonly{---} \\
\code{double elapsedTime( const unsigned phase ) }

\item[Cv] \texonly{---} \\
\code{double Cv_sqr() }

\end{description}

\C Local Variables: 
\C mode: latex
\C TeX-master: "lqns"
\C End: 

